% generated from JIRA project LVV
% using template at /usr/local/lib/python3.7/site-packages/docsteady/templates/dm-tpr.latex.jinja2.
% using docsteady version 1.2rc21
% Please do not edit -- update information in Jira instead

\documentclass[DM,lsstdraft,STR,toc]{lsstdoc}
\usepackage{geometry}
\usepackage{longtable,booktabs}
\usepackage{enumitem}
\usepackage{arydshln}
\usepackage{attachfile}
\usepackage{array}

\newcolumntype{L}[1]{>{\raggedright\let\newline\\\arraybackslash\hspace{0pt}}p{#1}}

\input meta.tex

\newcommand{\attachmentsUrl}{https://github.com/\gitorg/\lsstDocType-\lsstDocNum/blob/\gitref/attachments}
\providecommand{\tightlist}{
  \setlength{\itemsep}{0pt}\setlength{\parskip}{0pt}}

\setcounter{tocdepth}{4}

\begin{document}

\def\milestoneName{Science Pipelines Release 20.0.0 Acceptance Test Campaign}
\def\milestoneId{LVV-P71}
\def\product{Acceptance}

\setDocCompact{true}

\title{LVV-P71: Science Pipelines Release 20.0.0 Acceptance Test Campaign Test Plan and Report}
\setDocRef{\lsstDocType-\lsstDocNum}
\date{\vcsdate}
\author{ Jeffrey Carlin }

% Most recent last
\setDocChangeRecord{
\addtohist{}{2020-06-24}{First draft}{Jeff Carlin}
\addtohist{1.0}{2020-08-05}{Test plan LVV-P71 approved, test activity ready to start. \jira{DM-25646}}{Jeff Carlin}
\addtohist{2.0}{2020-11-16}{Test campaign LVV-P71 completed and results approved. \jira{DM-26364}}{Jeff Carlin}
}

\setDocCurator{Jeff Carlin}
\setDocUpstreamLocation{\url{https://github.com/lsst-dm/\lsstDocType-\lsstDocNum}}
\setDocUpstreamVersion{\vcsrevision}



\setDocAbstract{
This is the test plan and report for LVV-P71 (Science Pipelines Release 20.0.0 Acceptance Test Campaign),
an LSST milestone pertaining to the Data Management Subsystem.
}


\maketitle

\section{Introduction}
\label{sect:intro}


\subsection{Objectives}
\label{sect:objectives}

 This Acceptance Test campaign aims to verify a subset of
\href{https://lse-61.lsst.io/}{DMSR} (\citeds{LSE-61}) requirements related to
the LSST Science Pipelines. It will be executed in conjunction with the
release of Science Pipelines Version 20.0.0, but the pipeline release is
not contingent upon this test campaign. This Test Plan aims to
demonstrate that the included requirements have been met by Version
20.0.0 of the Pipelines, and to thus fully verify their completion and
readiness for LSST Operations.



\subsection{System Overview}
\label{sect:systemoverview}

 The tests to be executed are intended to verify that the DM system
satisfies a subset of the requirements outlined in the Data Management
System Requirements (DMSR; \href{https://lse-61.lsst.io/}{LSE-61} ).
This subset of requirements is related to pipeline algorithms, and was
selected for this campaign to coincide with the release of a new version
of the LSST Science Pipelines. Additional DMSR requirements will be
verified in later Acceptance Test
Campaigns.\\[2\baselineskip]\textbf{Applicable
Documents:}\\[2\baselineskip]\citeds{LSE-61} Data Management System
Requirements\\
\citeds{LDM-503} Data Management Test Plan\\
\citeds{LDM-639} LSST Data Management Acceptance Test Specification (issue
2.1)\\[2\baselineskip]The tests will be performed using the HSC-RC2
dataset (as defined in
\href{https://jira.lsstcorp.org/browse/DM-11345}{DM-11345} ). When
possible, we will start our tests with the data products resulting from
processing HSC-RC2 with the v20\_0\_0\_rc1 pipelines release candidate
(DM-24478) that was used to create v20 of the Science Pipelines.


\subsection{Document Overview}
\label{sect:docoverview}

This document was generated from Jira, obtaining the relevant information from the
\href{https://jira.lsstcorp.org/secure/Tests.jspa\#/testPlan/LVV-P71}{LVV-P71}
~Jira Test Plan and related Test Cycles (
\href{https://jira.lsstcorp.org/secure/Tests.jspa\#/testCycle/LVV-C153}{LVV-C153}
).

Section \ref{sect:intro} provides an overview of the test campaign, the system under test (\product{}),
the applicable documentation, and explains how this document is organized.
Section \ref{sect:testplan} provides additional information about the test plan, like for example the configuration
used for this test or related documentation.
Section \ref{sect:personnel} describes the necessary roles and lists the individuals assigned to them.

Section \ref{sect:overview} provides a summary of the test results, including an overview in Table \ref{table:summary},
an overall assessment statement and suggestions for possible improvements.
Section \ref{sect:detailedtestresults} provides detailed results for each step in each test case.

The current status of test plan \href{https://jira.lsstcorp.org/secure/Tests.jspa\#/testPlan/LVV-P71}{LVV-P71} in Jira is \textbf{ Approved }.

\subsection{References}
\label{sect:references}
\renewcommand{\refname}{}
\bibliography{lsst,refs,books,refs_ads,local}


\newpage
\section{Test Plan Details}
\label{sect:testplan}


\subsection{Data Collection}

  Observing is not required for this test campaign.

\subsection{Verification Environment}
\label{sect:hwconf}
  The ``lsst-lsp-stable'' instance of the LSST Science Platform (LSP),
hosted at the LDF, and the ``lsst-dev'' development cluster at NCSA. In
particular, we will use Release 20.0.0 of the Pipelines.

  \subsection{Entry Criteria}
  Release and availability of Science Pipelines version 20.



\subsection{Related Documentation}

The documentation related to this test campaign should be provided in the following DocuShare Collection
(as per Verification Artifacts in Jira test plan LVV-P71).

\begin{itemize}
\item DocuShare Collection Not Specified
\end{itemize}



\subsection{PMCS Activity}

Primavera milestones related to the test campaign.

\begin{itemize}
\item None
\end{itemize}


\newpage
\section{Personnel}
\label{sect:personnel}

The personnel involved in the test campaign is shown in the following table.

{\small
\begin{longtable}{p{3cm}p{3cm}p{3cm}p{6cm}}
\hline
\multicolumn{2}{r}{T. Plan \href{https://jira.lsstcorp.org/secure/Tests.jspa\#/testPlan/LVV-P71}{LVV-P71} owner:} &
\multicolumn{2}{l}{\textbf{ Jeffrey Carlin } }\\\hline
\multicolumn{2}{r}{T. Cycle \href{https://jira.lsstcorp.org/secure/Tests.jspa\#/testCycle/LVV-C153}{LVV-C153} owner:} &
\multicolumn{2}{l}{\textbf{
Jeffrey Carlin }
} \\\hline
\textbf{Test Cases} & \textbf{Assigned to} & \textbf{Executed by} & \textbf{Additional Test Personnel} \\ \hline
\href{https://jira.lsstcorp.org/secure/Tests.jspa#/testCase/LVV-T28}{LVV-T28}
& {\small Colin Slater } & {\small  } &
\begin{minipage}[]{6cm}
\smallskip
{\small  }
\medskip
\end{minipage}
\\ \hline
\href{https://jira.lsstcorp.org/secure/Tests.jspa#/testCase/LVV-T133}{LVV-T133}
& {\small Robert Lupton } & {\small  } &
\begin{minipage}[]{6cm}
\smallskip
{\small  }
\medskip
\end{minipage}
\\ \hline
\href{https://jira.lsstcorp.org/secure/Tests.jspa#/testCase/LVV-T1087}{LVV-T1087}
& {\small Fritz Mueller } & {\small  } &
\begin{minipage}[]{6cm}
\smallskip
{\small  }
\medskip
\end{minipage}
\\ \hline
\href{https://jira.lsstcorp.org/secure/Tests.jspa#/testCase/LVV-T1086}{LVV-T1086}
& {\small Fritz Mueller } & {\small  } &
\begin{minipage}[]{6cm}
\smallskip
{\small  }
\medskip
\end{minipage}
\\ \hline
\href{https://jira.lsstcorp.org/secure/Tests.jspa#/testCase/LVV-T1085}{LVV-T1085}
& {\small Fritz Mueller } & {\small  } &
\begin{minipage}[]{6cm}
\smallskip
{\small  }
\medskip
\end{minipage}
\\ \hline
\href{https://jira.lsstcorp.org/secure/Tests.jspa#/testCase/LVV-T1232}{LVV-T1232}
& {\small Colin Slater } & {\small  } &
\begin{minipage}[]{6cm}
\smallskip
{\small  }
\medskip
\end{minipage}
\\ \hline
\href{https://jira.lsstcorp.org/secure/Tests.jspa#/testCase/LVV-T40}{LVV-T40}
& {\small Jim Bosch } & {\small  } &
\begin{minipage}[]{6cm}
\smallskip
{\small  }
\medskip
\end{minipage}
\\ \hline
\href{https://jira.lsstcorp.org/secure/Tests.jspa#/testCase/LVV-T1759}{LVV-T1759}
& {\small Jeffrey Carlin } & {\small  } &
\begin{minipage}[]{6cm}
\smallskip
{\small  }
\medskip
\end{minipage}
\\ \hline
\href{https://jira.lsstcorp.org/secure/Tests.jspa#/testCase/LVV-T1758}{LVV-T1758}
& {\small Jeffrey Carlin } & {\small  } &
\begin{minipage}[]{6cm}
\smallskip
{\small  }
\medskip
\end{minipage}
\\ \hline
\href{https://jira.lsstcorp.org/secure/Tests.jspa#/testCase/LVV-T1756}{LVV-T1756}
& {\small Jeffrey Carlin } & {\small  } &
\begin{minipage}[]{6cm}
\smallskip
{\small  }
\medskip
\end{minipage}
\\ \hline
\href{https://jira.lsstcorp.org/secure/Tests.jspa#/testCase/LVV-T1757}{LVV-T1757}
& {\small Jeffrey Carlin } & {\small  } &
\begin{minipage}[]{6cm}
\smallskip
{\small  }
\medskip
\end{minipage}
\\ \hline
\href{https://jira.lsstcorp.org/secure/Tests.jspa#/testCase/LVV-T125}{LVV-T125}
& {\small Robert Lupton } & {\small  } &
\begin{minipage}[]{6cm}
\smallskip
{\small  }
\medskip
\end{minipage}
\\ \hline
\href{https://jira.lsstcorp.org/secure/Tests.jspa#/testCase/LVV-T36}{LVV-T36}
& {\small Eric Bellm } & {\small  } &
\begin{minipage}[]{6cm}
\smallskip
{\small  }
\medskip
\end{minipage}
\\ \hline
\href{https://jira.lsstcorp.org/secure/Tests.jspa#/testCase/LVV-T126}{LVV-T126}
& {\small Eric Bellm } & {\small  } &
\begin{minipage}[]{6cm}
\smallskip
{\small  }
\medskip
\end{minipage}
\\ \hline
\href{https://jira.lsstcorp.org/secure/Tests.jspa#/testCase/LVV-T39}{LVV-T39}
& {\small Jim Bosch } & {\small  } &
\begin{minipage}[]{6cm}
\smallskip
{\small  }
\medskip
\end{minipage}
\\ \hline
\href{https://jira.lsstcorp.org/secure/Tests.jspa#/testCase/LVV-T46}{LVV-T46}
& {\small Eric Bellm } & {\small  } &
\begin{minipage}[]{6cm}
\smallskip
{\small  }
\medskip
\end{minipage}
\\ \hline
\href{https://jira.lsstcorp.org/secure/Tests.jspa#/testCase/LVV-T38}{LVV-T38}
& {\small Eric Bellm } & {\small  } &
\begin{minipage}[]{6cm}
\smallskip
{\small  }
\medskip
\end{minipage}
\\ \hline
\href{https://jira.lsstcorp.org/secure/Tests.jspa#/testCase/LVV-T42}{LVV-T42}
& {\small Jim Bosch } & {\small  } &
\begin{minipage}[]{6cm}
\smallskip
{\small  }
\medskip
\end{minipage}
\\ \hline
\href{https://jira.lsstcorp.org/secure/Tests.jspa#/testCase/LVV-T149}{LVV-T149}
& {\small Colin Slater } & {\small  } &
\begin{minipage}[]{6cm}
\smallskip
{\small  }
\medskip
\end{minipage}
\\ \hline
\href{https://jira.lsstcorp.org/secure/Tests.jspa#/testCase/LVV-T151}{LVV-T151}
& {\small Colin Slater } & {\small  } &
\begin{minipage}[]{6cm}
\smallskip
{\small  }
\medskip
\end{minipage}
\\ \hline
\href{https://jira.lsstcorp.org/secure/Tests.jspa#/testCase/LVV-T45}{LVV-T45}
& {\small Eric Bellm } & {\small  } &
\begin{minipage}[]{6cm}
\smallskip
{\small  }
\medskip
\end{minipage}
\\ \hline
\href{https://jira.lsstcorp.org/secure/Tests.jspa#/testCase/LVV-T146}{LVV-T146}
& {\small Robert Gruendl } & {\small  } &
\begin{minipage}[]{6cm}
\smallskip
{\small  }
\medskip
\end{minipage}
\\ \hline
\href{https://jira.lsstcorp.org/secure/Tests.jspa#/testCase/LVV-T144}{LVV-T144}
& {\small Kian-Tat Lim } & {\small  } &
\begin{minipage}[]{6cm}
\smallskip
{\small  }
\medskip
\end{minipage}
\\ \hline
\href{https://jira.lsstcorp.org/secure/Tests.jspa#/testCase/LVV-T145}{LVV-T145}
& {\small Robert Lupton } & {\small  } &
\begin{minipage}[]{6cm}
\smallskip
{\small  }
\medskip
\end{minipage}
\\ \hline
\href{https://jira.lsstcorp.org/secure/Tests.jspa#/testCase/LVV-T1264}{LVV-T1264}
& {\small Robert Gruendl } & {\small  } &
\begin{minipage}[]{6cm}
\smallskip
{\small  }
\medskip
\end{minipage}
\\ \hline
\end{longtable}
}

\newpage

\section{Test Campaign Overview}
\label{sect:overview}

\subsection{Summary}
\label{sect:summarytable}

{\small
\begin{longtable}{p{2cm}cp{2.3cm}p{8.6cm}p{2.3cm}}
\toprule
\multicolumn{2}{r}{ T. Plan \href{https://jira.lsstcorp.org/secure/Tests.jspa\#/testPlan/LVV-P71}{LVV-P71}:} &
\multicolumn{2}{p{10.9cm}}{\textbf{ Science Pipelines Release 20.0.0 Acceptance Test Campaign }} & Approved \\\hline
\multicolumn{2}{r}{ T. Cycle \href{https://jira.lsstcorp.org/secure/Tests.jspa\#/testCycle/LVV-C153}{LVV-C153}:} &
\multicolumn{2}{p{10.9cm}}{\textbf{ Pipelines v20 Release DM Acceptance Test Campaign }} & Not Executed \\\hline
\textbf{Test Cases} &  \textbf{Ver.} & \textbf{Status} & \textbf{Comment} & \textbf{Issues} \\\toprule
\href{https://jira.lsstcorp.org/secure/Tests.jspa#/testCase/LVV-T28}{LVV-T28}
&  1
& Not Executed &
\begin{minipage}[]{9cm}
\smallskip

\medskip
\end{minipage}
&
\\\hline
\href{https://jira.lsstcorp.org/secure/Tests.jspa#/testCase/LVV-T133}{LVV-T133}
&  1
& Not Executed &
\begin{minipage}[]{9cm}
\smallskip

\medskip
\end{minipage}
&
\\\hline
\href{https://jira.lsstcorp.org/secure/Tests.jspa#/testCase/LVV-T1087}{LVV-T1087}
&  1
& Not Executed &
\begin{minipage}[]{9cm}
\smallskip

\medskip
\end{minipage}
&
\\\hline
\href{https://jira.lsstcorp.org/secure/Tests.jspa#/testCase/LVV-T1086}{LVV-T1086}
&  1
& Not Executed &
\begin{minipage}[]{9cm}
\smallskip

\medskip
\end{minipage}
&
\\\hline
\href{https://jira.lsstcorp.org/secure/Tests.jspa#/testCase/LVV-T1085}{LVV-T1085}
&  1
& Not Executed &
\begin{minipage}[]{9cm}
\smallskip

\medskip
\end{minipage}
&
\\\hline
\href{https://jira.lsstcorp.org/secure/Tests.jspa#/testCase/LVV-T1232}{LVV-T1232}
&  1
& Not Executed &
\begin{minipage}[]{9cm}
\smallskip

\medskip
\end{minipage}
&
\\\hline
\href{https://jira.lsstcorp.org/secure/Tests.jspa#/testCase/LVV-T40}{LVV-T40}
&  1
& Not Executed &
\begin{minipage}[]{9cm}
\smallskip

\medskip
\end{minipage}
&
\\\hline
\href{https://jira.lsstcorp.org/secure/Tests.jspa#/testCase/LVV-T1759}{LVV-T1759}
&  1
& Not Executed &
\begin{minipage}[]{9cm}
\smallskip

\medskip
\end{minipage}
&
\\\hline
\href{https://jira.lsstcorp.org/secure/Tests.jspa#/testCase/LVV-T1758}{LVV-T1758}
&  1
& Not Executed &
\begin{minipage}[]{9cm}
\smallskip

\medskip
\end{minipage}
&
\\\hline
\href{https://jira.lsstcorp.org/secure/Tests.jspa#/testCase/LVV-T1756}{LVV-T1756}
&  1
& Not Executed &
\begin{minipage}[]{9cm}
\smallskip

\medskip
\end{minipage}
&
\\\hline
\href{https://jira.lsstcorp.org/secure/Tests.jspa#/testCase/LVV-T1757}{LVV-T1757}
&  1
& Not Executed &
\begin{minipage}[]{9cm}
\smallskip

\medskip
\end{minipage}
&
\\\hline
\href{https://jira.lsstcorp.org/secure/Tests.jspa#/testCase/LVV-T125}{LVV-T125}
&  1
& Not Executed &
\begin{minipage}[]{9cm}
\smallskip

\medskip
\end{minipage}
&
\\\hline
\href{https://jira.lsstcorp.org/secure/Tests.jspa#/testCase/LVV-T36}{LVV-T36}
&  1
& Not Executed &
\begin{minipage}[]{9cm}
\smallskip

\medskip
\end{minipage}
&
\\\hline
\href{https://jira.lsstcorp.org/secure/Tests.jspa#/testCase/LVV-T126}{LVV-T126}
&  1
& Not Executed &
\begin{minipage}[]{9cm}
\smallskip

\medskip
\end{minipage}
&
\\\hline
\href{https://jira.lsstcorp.org/secure/Tests.jspa#/testCase/LVV-T39}{LVV-T39}
&  1
& Not Executed &
\begin{minipage}[]{9cm}
\smallskip

\medskip
\end{minipage}
&
\\\hline
\href{https://jira.lsstcorp.org/secure/Tests.jspa#/testCase/LVV-T46}{LVV-T46}
&  1
& Not Executed &
\begin{minipage}[]{9cm}
\smallskip

\medskip
\end{minipage}
&
\\\hline
\href{https://jira.lsstcorp.org/secure/Tests.jspa#/testCase/LVV-T38}{LVV-T38}
&  1
& Not Executed &
\begin{minipage}[]{9cm}
\smallskip

\medskip
\end{minipage}
&
\\\hline
\href{https://jira.lsstcorp.org/secure/Tests.jspa#/testCase/LVV-T42}{LVV-T42}
&  1
& Not Executed &
\begin{minipage}[]{9cm}
\smallskip

\medskip
\end{minipage}
&
\\\hline
\href{https://jira.lsstcorp.org/secure/Tests.jspa#/testCase/LVV-T149}{LVV-T149}
&  1
& Not Executed &
\begin{minipage}[]{9cm}
\smallskip

\medskip
\end{minipage}
&
\\\hline
\href{https://jira.lsstcorp.org/secure/Tests.jspa#/testCase/LVV-T151}{LVV-T151}
&  1
& Not Executed &
\begin{minipage}[]{9cm}
\smallskip

\medskip
\end{minipage}
&
\\\hline
\href{https://jira.lsstcorp.org/secure/Tests.jspa#/testCase/LVV-T45}{LVV-T45}
&  1
& Not Executed &
\begin{minipage}[]{9cm}
\smallskip

\medskip
\end{minipage}
&
\\\hline
\href{https://jira.lsstcorp.org/secure/Tests.jspa#/testCase/LVV-T146}{LVV-T146}
&  1
& Not Executed &
\begin{minipage}[]{9cm}
\smallskip

\medskip
\end{minipage}
&
\\\hline
\href{https://jira.lsstcorp.org/secure/Tests.jspa#/testCase/LVV-T144}{LVV-T144}
&  1
& Not Executed &
\begin{minipage}[]{9cm}
\smallskip

\medskip
\end{minipage}
&
\\\hline
\href{https://jira.lsstcorp.org/secure/Tests.jspa#/testCase/LVV-T145}{LVV-T145}
&  1
& Not Executed &
\begin{minipage}[]{9cm}
\smallskip

\medskip
\end{minipage}
&
\\\hline
\href{https://jira.lsstcorp.org/secure/Tests.jspa#/testCase/LVV-T1264}{LVV-T1264}
&  1
& Not Executed &
\begin{minipage}[]{9cm}
\smallskip

\medskip
\end{minipage}
&
\\\hline
\caption{Test Campaign Summary}
\label{table:summary}
\end{longtable}
}

\subsection{Overall Assessment}
\label{sect:overallassessment}

Not yet available.

\subsection{Recommended Improvements}
\label{sect:recommendations}

Not yet available.

\newpage
\section{Detailed Test Results}
\label{sect:detailedtestresults}

\subsection{Test Cycle LVV-C153 }

Open test cycle {\it \href{https://jira.lsstcorp.org/secure/Tests.jspa#/testrun/LVV-C153}{Pipelines v20 Release DM Acceptance Test Campaign}} in Jira.

Test Cycle name: Pipelines v20 Release DM Acceptance Test Campaign\\
Status: Not Executed

This test cycle verifies a subset of
\href{https://lse-61.lsst.io/}{DMSR} (\citeds{LSE-61}) requirements related to
the LSST Science Pipelines, in order to verify their completion and
readiness for LSST Operations (i.e., that the requirements laid out in
\citeds{LSE-61} have been met by the DM Systems).

\subsubsection{Software Version/Baseline}
All tests will be performed with LSST Science Pipelines release version
20.0.0, including its algorithms and resulting science data products.

\subsubsection{Configuration}
Not provided.

\subsubsection{Test Cases in LVV-C153 Test Cycle}

\paragraph{ LVV-T28 - Verify implementation of Measurements in catalogs }\mbox{}\\

Version \textbf{1}.
Open  \href{https://jira.lsstcorp.org/secure/Tests.jspa#/testCase/LVV-T28}{\textit{ LVV-T28 } }
test case in Jira.

Verify that source measurements in catalogs are in flux units.

\textbf{ Preconditions}:\\


Execution status: {\bf Not Executed }

Final comment:\\


Detailed steps results:

\begin{longtable}{p{1cm}p{15cm}}
\hline
{Step} & Step Details\\ \hline
1 & Description \\
 & \begin{minipage}[t]{15cm}
{\footnotesize
Identify the path to the data repository, which we will refer to as
`DATA/path', then execute the following:

\medskip }
\end{minipage}
\\ \cdashline{2-2}

 & Example Code \\
 & \begin{minipage}[t]{15cm}{\footnotesize
\begin{verbatim}
import lsst.daf.persistence as dafPersist
butler = dafPersist.Butler(inputs='DATA/path')
\end{verbatim}

\medskip }
\end{minipage} \\ \cdashline{2-2}

 & Expected Result \\
 & \begin{minipage}[t]{15cm}{\footnotesize
Butler repo available for reading.

\medskip }
\end{minipage} \\ \cdashline{2-2}

 & Actual Result \\
 & \begin{minipage}[t]{15cm}{\footnotesize

\medskip }
\end{minipage} \\ \cdashline{2-2}

 & Status: \textbf{ Not Executed } \\ \hline

2 & Description \\
 & \begin{minipage}[t]{15cm}
{\footnotesize
Identify and read appropriate processed precursor datasets with the
Butler, including one containing single-visit images, one with coadds,
and one with difference imaging.~

\medskip }
\end{minipage}
\\ \cdashline{2-2}


 & Expected Result \\
 & \begin{minipage}[t]{15cm}{\footnotesize

\medskip }
\end{minipage} \\ \cdashline{2-2}

 & Actual Result \\
 & \begin{minipage}[t]{15cm}{\footnotesize

\medskip }
\end{minipage} \\ \cdashline{2-2}

 & Status: \textbf{ Not Executed } \\ \hline

3 & Description \\
 & \begin{minipage}[t]{15cm}
{\footnotesize
Verify that each of the single-visit, coadd, and difference image
catalogs provide measurements in flux units.

\medskip }
\end{minipage}
\\ \cdashline{2-2}


 & Expected Result \\
 & \begin{minipage}[t]{15cm}{\footnotesize
Confirmation of measurements in catalogs encoded in flux units.

\medskip }
\end{minipage} \\ \cdashline{2-2}

 & Actual Result \\
 & \begin{minipage}[t]{15cm}{\footnotesize

\medskip }
\end{minipage} \\ \cdashline{2-2}

 & Status: \textbf{ Not Executed } \\ \hline

\end{longtable}

\paragraph{ LVV-T133 - Verify implementation of Provide Beam Projector Coordinate Calculation
Software }\mbox{}\\

Version \textbf{1}.
Open  \href{https://jira.lsstcorp.org/secure/Tests.jspa#/testCase/LVV-T133}{\textit{ LVV-T133 } }
test case in Jira.

Verify that the DMS provides software to calculate coordinates relating
the collimated beam projector position and telescope pupil position to
the illumination position on the telescope optical elements and focal
plane.

\textbf{ Preconditions}:\\


Execution status: {\bf Not Executed }

Final comment:\\


Detailed steps results:

\begin{longtable}{p{1cm}p{15cm}}
\hline
{Step} & Step Details\\ \hline
1 & Description \\
 & \begin{minipage}[t]{15cm}
{\footnotesize
On the LSST development cluster or notebook aspect, git clone the repo
containing the CBP package: \url{https://github.com/lsst/cbp}

\medskip }
\end{minipage}
\\ \cdashline{2-2}


 & Expected Result \\
 & \begin{minipage}[t]{15cm}{\footnotesize

\medskip }
\end{minipage} \\ \cdashline{2-2}

 & Actual Result \\
 & \begin{minipage}[t]{15cm}{\footnotesize

\medskip }
\end{minipage} \\ \cdashline{2-2}

 & Status: \textbf{ Not Executed } \\ \hline

2 & Description \\
 & \begin{minipage}[t]{15cm}
{\footnotesize
Follow the steps in the package README to install the package.

\medskip }
\end{minipage}
\\ \cdashline{2-2}


 & Expected Result \\
 & \begin{minipage}[t]{15cm}{\footnotesize

\medskip }
\end{minipage} \\ \cdashline{2-2}

 & Actual Result \\
 & \begin{minipage}[t]{15cm}{\footnotesize

\medskip }
\end{minipage} \\ \cdashline{2-2}

 & Status: \textbf{ Not Executed } \\ \hline

3 & Description \\
 & \begin{minipage}[t]{15cm}
{\footnotesize
Confirm that the package can be loaded in python, and that some of the
tests in the `tests/` folder will execute.

\medskip }
\end{minipage}
\\ \cdashline{2-2}


 & Expected Result \\
 & \begin{minipage}[t]{15cm}{\footnotesize
Successful execution of test scripts, which demonstrate the calculation
of beam projector coordinates.

\medskip }
\end{minipage} \\ \cdashline{2-2}

 & Actual Result \\
 & \begin{minipage}[t]{15cm}{\footnotesize

\medskip }
\end{minipage} \\ \cdashline{2-2}

 & Status: \textbf{ Not Executed } \\ \hline

\end{longtable}

\paragraph{ LVV-T1087 - Full Table Joins Functional Test }\mbox{}\\

Version \textbf{1}.
Open  \href{https://jira.lsstcorp.org/secure/Tests.jspa#/testCase/LVV-T1087}{\textit{ LVV-T1087 } }
test case in Jira.

The objective of this test is to ensure that the full table join queries
are performing as expected and establish a timing baseline benchmark for
these types of queries.

\textbf{ Preconditions}:\\
QSERV has been set-up following procedure at ~LVV-T1017.

Execution status: {\bf Not Executed }

Final comment:\\


Detailed steps results:

\begin{longtable}{p{1cm}p{15cm}}
\hline
{Step} & Step Details\\ \hline
1 & Description \\
 & \begin{minipage}[t]{15cm}
{\footnotesize
Execute query:\\[2\baselineskip]\textbf{SELECT} o.deepSourceId,
s.objectId, s.id, o.ra, o.decl\\
\textbf{~ ~ FROM} Object o, Source s WHERE o.deepSourceId=s.objectId\\
\hspace*{0.333em} ~ \textbf{AND} s . flux\_sinc \textbf{BETWEEN} 0.3
\textbf{AND} 0.31\\[2\baselineskip]and record execution time.

\medskip }
\end{minipage}
\\ \cdashline{2-2}


 & Expected Result \\
 & \begin{minipage}[t]{15cm}{\footnotesize
Query expected to run in less than 12 hours.

\medskip }
\end{minipage} \\ \cdashline{2-2}

 & Actual Result \\
 & \begin{minipage}[t]{15cm}{\footnotesize

\medskip }
\end{minipage} \\ \cdashline{2-2}

 & Status: \textbf{ Not Executed } \\ \hline

2 & Description \\
 & \begin{minipage}[t]{15cm}
{\footnotesize
Execute query:\\[2\baselineskip]\textbf{SELECT} o.deepSourceId,
f.psfFlux \textbf{FROM} Object o, ForcedSource f\\
\textbf{~ ~ WHERE} o.deepSourceId=f.deepSourceId\\
\textbf{~ ~ AND} f . psfFlux \textbf{BETWEEN} 0.13 \textbf{AND}
0.14\\[2\baselineskip]and record execution time.

\medskip }
\end{minipage}
\\ \cdashline{2-2}


 & Expected Result \\
 & \begin{minipage}[t]{15cm}{\footnotesize
Query expected to run in less than 12 hours.

\medskip }
\end{minipage} \\ \cdashline{2-2}

 & Actual Result \\
 & \begin{minipage}[t]{15cm}{\footnotesize

\medskip }
\end{minipage} \\ \cdashline{2-2}

 & Status: \textbf{ Not Executed } \\ \hline

\end{longtable}

\paragraph{ LVV-T1086 - Full Table Scans Functional Test }\mbox{}\\

Version \textbf{1}.
Open  \href{https://jira.lsstcorp.org/secure/Tests.jspa#/testCase/LVV-T1086}{\textit{ LVV-T1086 } }
test case in Jira.

The objective of this test is to ensure that the full table scan queries
are performing as expected and establish a timing baseline benchmark for
these types of queries.

\textbf{ Preconditions}:\\
QSERV has been set-up following procedure at ~LVV-T1017.

Execution status: {\bf Not Executed }

Final comment:\\


Detailed steps results:

\begin{longtable}{p{1cm}p{15cm}}
\hline
{Step} & Step Details\\ \hline
1 & Description \\
 & \begin{minipage}[t]{15cm}
{\footnotesize
Execute query:\\[2\baselineskip]\textbf{SELECT} ra , decl , u\_psfFlux ,
g\_psfFlux , r\_psfFlux \textbf{FROM} Object\\
\textbf{WHERE} y\_shapeIxx \textbf{BETWEEN} 20 \textbf{AND}
20.1\\[3\baselineskip]and record execution time and output size.

\medskip }
\end{minipage}
\\ \cdashline{2-2}


 & Expected Result \\
 & \begin{minipage}[t]{15cm}{\footnotesize
Query expected to run in less than 1 hour.\\[2\baselineskip]

\medskip }
\end{minipage} \\ \cdashline{2-2}

 & Actual Result \\
 & \begin{minipage}[t]{15cm}{\footnotesize

\medskip }
\end{minipage} \\ \cdashline{2-2}

 & Status: \textbf{ Not Executed } \\ \hline

2 & Description \\
 & \begin{minipage}[t]{15cm}
{\footnotesize
Execute query:\\[2\baselineskip]\textbf{SELECT} COUNT(*) \textbf{FROM}
Source \textbf{WHERE} flux\_sinc \textbf{BETWEEN} 1 \textbf{AND}
1.1\\[2\baselineskip]and record the execution time

\medskip }
\end{minipage}
\\ \cdashline{2-2}


 & Expected Result \\
 & \begin{minipage}[t]{15cm}{\footnotesize
Query expected to run in less than 12 hours.

\medskip }
\end{minipage} \\ \cdashline{2-2}

 & Actual Result \\
 & \begin{minipage}[t]{15cm}{\footnotesize

\medskip }
\end{minipage} \\ \cdashline{2-2}

 & Status: \textbf{ Not Executed } \\ \hline

3 & Description \\
 & \begin{minipage}[t]{15cm}
{\footnotesize
Execute query:\\[2\baselineskip]\textbf{SELECT} COUNT(*) \textbf{FROM}
ForcedSource \textbf{WHERE} psfFlux \textbf{BETWEEN} 0.1 \textbf{AND}
0.2\\[2\baselineskip]and record the execution time

\medskip }
\end{minipage}
\\ \cdashline{2-2}


 & Expected Result \\
 & \begin{minipage}[t]{15cm}{\footnotesize
Query expected to run in less than 12 hours.

\medskip }
\end{minipage} \\ \cdashline{2-2}

 & Actual Result \\
 & \begin{minipage}[t]{15cm}{\footnotesize

\medskip }
\end{minipage} \\ \cdashline{2-2}

 & Status: \textbf{ Not Executed } \\ \hline

\end{longtable}

\paragraph{ LVV-T1085 - Short Queries Functional Test }\mbox{}\\

Version \textbf{1}.
Open  \href{https://jira.lsstcorp.org/secure/Tests.jspa#/testCase/LVV-T1085}{\textit{ LVV-T1085 } }
test case in Jira.

The objective of this test is to ensure that the short queries are
performing as expected and establish a timing baseline benchmark for
these types of queries.

\textbf{ Preconditions}:\\
QSERV has been set-up following procedure at ~LVV-T1017.

Execution status: {\bf Not Executed }

Final comment:\\


Detailed steps results:

\begin{longtable}{p{1cm}p{15cm}}
\hline
{Step} & Step Details\\ \hline
1 & Description \\
 & \begin{minipage}[t]{15cm}
{\footnotesize
Execute single object selection:\\[2\baselineskip]\textbf{SELECT} *
\textbf{FROM} Object~\textbf{WHERE} deepSourceId =
9292041530376264\\[2\baselineskip]and record execution time.

\medskip }
\end{minipage}
\\ \cdashline{2-2}


 & Expected Result \\
 & \begin{minipage}[t]{15cm}{\footnotesize
Query runs in less than 10 seconds.

\medskip }
\end{minipage} \\ \cdashline{2-2}

 & Actual Result \\
 & \begin{minipage}[t]{15cm}{\footnotesize

\medskip }
\end{minipage} \\ \cdashline{2-2}

 & Status: \textbf{ Not Executed } \\ \hline

2 & Description \\
 & \begin{minipage}[t]{15cm}
{\footnotesize
Execute spatial area selection from
Object:\\[2\baselineskip]\textbf{SELECT COUNT(*)} \textbf{FROM} Object
\textbf{WHERE}~\\

~qserv\_areaspec\_box(316.582327, −6.839078, 316.653938, −6.781822)

and record execution time.

\medskip }
\end{minipage}
\\ \cdashline{2-2}


 & Expected Result \\
 & \begin{minipage}[t]{15cm}{\footnotesize
Query runs in less than 10 seconds.

\medskip }
\end{minipage} \\ \cdashline{2-2}

 & Actual Result \\
 & \begin{minipage}[t]{15cm}{\footnotesize

\medskip }
\end{minipage} \\ \cdashline{2-2}

 & Status: \textbf{ Not Executed } \\ \hline

\end{longtable}

\paragraph{ LVV-T1232 - Verify Implementation of Catalog Export Formats From the Portal Aspect }\mbox{}\\

Version \textbf{1}.
Open  \href{https://jira.lsstcorp.org/secure/Tests.jspa#/testCase/LVV-T1232}{\textit{ LVV-T1232 } }
test case in Jira.

Verify that catalog data is exportable from the portal aspect in a
variety of community-standard formats.

\textbf{ Preconditions}:\\


Execution status: {\bf Not Executed }

Final comment:\\


Detailed steps results:

\begin{longtable}{p{1cm}p{15cm}}
\hline
{Step} & Step Details\\ \hline
1 & Description \\
 & \begin{minipage}[t]{15cm}
{\footnotesize
Navigate to the Portal Aspect endpoint. ~The stable version should be
used for this test and is currently located at:
https://lsst-lsp-stable.ncsa.illinois.edu/portal/app/ .

\medskip }
\end{minipage}
\\ \cdashline{2-2}


 & Expected Result \\
 & \begin{minipage}[t]{15cm}{\footnotesize
A credential-entry screen should be displayed.

\medskip }
\end{minipage} \\ \cdashline{2-2}

 & Actual Result \\
 & \begin{minipage}[t]{15cm}{\footnotesize

\medskip }
\end{minipage} \\ \cdashline{2-2}

 & Status: \textbf{ Not Executed } \\ \hline

2 & Description \\
 & \begin{minipage}[t]{15cm}
{\footnotesize
Enter a valid set of credentials for an LSST user with LSP access on the
instance under test.

\medskip }
\end{minipage}
\\ \cdashline{2-2}


 & Expected Result \\
 & \begin{minipage}[t]{15cm}{\footnotesize
The Portal Aspect UI should be displayed following authentication.

\medskip }
\end{minipage} \\ \cdashline{2-2}

 & Actual Result \\
 & \begin{minipage}[t]{15cm}{\footnotesize

\medskip }
\end{minipage} \\ \cdashline{2-2}

 & Status: \textbf{ Not Executed } \\ \hline

3 & Description \\
 & \begin{minipage}[t]{15cm}
{\footnotesize
Select query type ``ADQL''.

\medskip }
\end{minipage}
\\ \cdashline{2-2}


 & Expected Result \\
 & \begin{minipage}[t]{15cm}{\footnotesize

\medskip }
\end{minipage} \\ \cdashline{2-2}

 & Actual Result \\
 & \begin{minipage}[t]{15cm}{\footnotesize

\medskip }
\end{minipage} \\ \cdashline{2-2}

 & Status: \textbf{ Not Executed } \\ \hline

4 & Description \\
 & \begin{minipage}[t]{15cm}
{\footnotesize
Execute the example query given in the example code below by entering
the text in the ADQL Query box, then clicking ``Search'' at the lower
left corner of the page.

\medskip }
\end{minipage}
\\ \cdashline{2-2}

 & Example Code \\
 & \begin{minipage}[t]{15cm}{\footnotesize
SELECT cntr, ra, decl, w1mpro\_ep, w2mpro\_ep, w3mpro\_ep FROM
wise\_00.allwise\_p3as\_mep WHERE CONTAINS(POINT('ICRS', ra, decl),
CIRCLE('ICRS', 192.85, 27.13, .2)) = 1

\medskip }
\end{minipage} \\ \cdashline{2-2}

 & Expected Result \\
 & \begin{minipage}[t]{15cm}{\footnotesize
A new page will load with the search results as a table, with some plots
as well.

\medskip }
\end{minipage} \\ \cdashline{2-2}

 & Actual Result \\
 & \begin{minipage}[t]{15cm}{\footnotesize

\medskip }
\end{minipage} \\ \cdashline{2-2}

 & Status: \textbf{ Not Executed } \\ \hline

5 & Description \\
 & \begin{minipage}[t]{15cm}
{\footnotesize
Click the icon that looks like a floppy disk (it says ``Save the content
as an IPAC, CSV, or TSV table'' when you mouse over it).

\medskip }
\end{minipage}
\\ \cdashline{2-2}


 & Expected Result \\
 & \begin{minipage}[t]{15cm}{\footnotesize

\medskip }
\end{minipage} \\ \cdashline{2-2}

 & Actual Result \\
 & \begin{minipage}[t]{15cm}{\footnotesize

\medskip }
\end{minipage} \\ \cdashline{2-2}

 & Status: \textbf{ Not Executed } \\ \hline

6 & Description \\
 & \begin{minipage}[t]{15cm}
{\footnotesize
\begin{itemize}
\tightlist
\item
  Select ``CSV'', then specify a destination to save the file on your
  local computer.
\item
  Select ``VOTable'', then specify a destination to save the file on
  your local computer.
\item
  Select ``FITS'', then specify a destination to save the file on your
  local computer.
\end{itemize}

\medskip }
\end{minipage}
\\ \cdashline{2-2}


 & Expected Result \\
 & \begin{minipage}[t]{15cm}{\footnotesize

\medskip }
\end{minipage} \\ \cdashline{2-2}

 & Actual Result \\
 & \begin{minipage}[t]{15cm}{\footnotesize

\medskip }
\end{minipage} \\ \cdashline{2-2}

 & Status: \textbf{ Not Executed } \\ \hline

7 & Description \\
 & \begin{minipage}[t]{15cm}
{\footnotesize
Open each of the files (either in TOPCAT, or using Astropy io tools).
Confirm that the data tables are well-formed, and that each table
contains the same columns and the same number of rows.

\medskip }
\end{minipage}
\\ \cdashline{2-2}


 & Expected Result \\
 & \begin{minipage}[t]{15cm}{\footnotesize

\medskip }
\end{minipage} \\ \cdashline{2-2}

 & Actual Result \\
 & \begin{minipage}[t]{15cm}{\footnotesize

\medskip }
\end{minipage} \\ \cdashline{2-2}

 & Status: \textbf{ Not Executed } \\ \hline

8 & Description \\
 & \begin{minipage}[t]{15cm}
{\footnotesize
Currently, there is no logout mechanism on the portal.\\
This should be updated as the system matures.\\[2\baselineskip]Simply
close the browser window.

\medskip }
\end{minipage}
\\ \cdashline{2-2}


 & Expected Result \\
 & \begin{minipage}[t]{15cm}{\footnotesize
Closed browser window. ~When navigating to the portal endpoint, expect
to execute the steps in LVV-T849.

\medskip }
\end{minipage} \\ \cdashline{2-2}

 & Actual Result \\
 & \begin{minipage}[t]{15cm}{\footnotesize

\medskip }
\end{minipage} \\ \cdashline{2-2}

 & Status: \textbf{ Not Executed } \\ \hline

\end{longtable}

\paragraph{ LVV-T40 - Verify implementation of Generate WCS for Visit Images }\mbox{}\\

Version \textbf{1}.
Open  \href{https://jira.lsstcorp.org/secure/Tests.jspa#/testCase/LVV-T40}{\textit{ LVV-T40 } }
test case in Jira.

Verify that Processed Visit Images produced by the AP and DRP pipelines
include FITS WCS accurate to specified \textbf{astrometricAccuracy} over
the bounds of the image.

\textbf{ Preconditions}:\\


Execution status: {\bf Not Executed }

Final comment:\\


Detailed steps results:

\begin{longtable}{p{1cm}p{15cm}}
\hline
{Step} & Step Details\\ \hline
1 & Description \\
 & \begin{minipage}[t]{15cm}
{\footnotesize
Identify an appropriate processed dataset for this test.

\medskip }
\end{minipage}
\\ \cdashline{2-2}


 & Expected Result \\
 & \begin{minipage}[t]{15cm}{\footnotesize
A dataset with Processed Visit Images available.

\medskip }
\end{minipage} \\ \cdashline{2-2}

 & Actual Result \\
 & \begin{minipage}[t]{15cm}{\footnotesize

\medskip }
\end{minipage} \\ \cdashline{2-2}

 & Status: \textbf{ Not Executed } \\ \hline

2 & Description \\
 & \begin{minipage}[t]{15cm}
{\footnotesize
Identify the path to the data repository, which we will refer to as
`DATA/path', then execute the following:

\medskip }
\end{minipage}
\\ \cdashline{2-2}

 & Example Code \\
 & \begin{minipage}[t]{15cm}{\footnotesize
\begin{verbatim}
import lsst.daf.persistence as dafPersist
butler = dafPersist.Butler(inputs='DATA/path')
\end{verbatim}

\medskip }
\end{minipage} \\ \cdashline{2-2}

 & Expected Result \\
 & \begin{minipage}[t]{15cm}{\footnotesize
Butler repo available for reading.

\medskip }
\end{minipage} \\ \cdashline{2-2}

 & Actual Result \\
 & \begin{minipage}[t]{15cm}{\footnotesize

\medskip }
\end{minipage} \\ \cdashline{2-2}

 & Status: \textbf{ Not Executed } \\ \hline

3 & Description \\
 & \begin{minipage}[t]{15cm}
{\footnotesize
Select a single visit from the dataset, and extract its WCS object and
the source list.

\medskip }
\end{minipage}
\\ \cdashline{2-2}


 & Expected Result \\
 & \begin{minipage}[t]{15cm}{\footnotesize
A table containing detected sources, and a WCS object associated with
that catalog.

\medskip }
\end{minipage} \\ \cdashline{2-2}

 & Actual Result \\
 & \begin{minipage}[t]{15cm}{\footnotesize

\medskip }
\end{minipage} \\ \cdashline{2-2}

 & Status: \textbf{ Not Executed } \\ \hline

4 & Description \\
 & \begin{minipage}[t]{15cm}
{\footnotesize
Confirm that each CCD within the visit image contains at
least~\textbf{astrometricMinStandards~}astrometric standards that were
used in deriving the astrometric solution.

\medskip }
\end{minipage}
\\ \cdashline{2-2}


 & Expected Result \\
 & \begin{minipage}[t]{15cm}{\footnotesize
At least \textbf{astrometricMinStandards} from each CCD\textbf{~}were
used in determining the WCS solution.

\medskip }
\end{minipage} \\ \cdashline{2-2}

 & Actual Result \\
 & \begin{minipage}[t]{15cm}{\footnotesize

\medskip }
\end{minipage} \\ \cdashline{2-2}

 & Status: \textbf{ Not Executed } \\ \hline

5 & Description \\
 & \begin{minipage}[t]{15cm}
{\footnotesize
Starting from the XY pixel coordinates of the sources, apply the WCS to
obtain RA, Dec coordinates.\\[2\baselineskip]

\medskip }
\end{minipage}
\\ \cdashline{2-2}


 & Expected Result \\
 & \begin{minipage}[t]{15cm}{\footnotesize
A list of RA, Dec coordinates for all sources in the catalog.

\medskip }
\end{minipage} \\ \cdashline{2-2}

 & Actual Result \\
 & \begin{minipage}[t]{15cm}{\footnotesize

\medskip }
\end{minipage} \\ \cdashline{2-2}

 & Status: \textbf{ Not Executed } \\ \hline

6 & Description \\
 & \begin{minipage}[t]{15cm}
{\footnotesize
We will assume that Gaia provides a source of ``truth.'' Match the
source list to Gaia DR2, and calculate the positional offset between the
test data and the Gaia catalog.

\medskip }
\end{minipage}
\\ \cdashline{2-2}


 & Expected Result \\
 & \begin{minipage}[t]{15cm}{\footnotesize
A matched catalog of sources in common between the test source list and
Gaia DR2.

\medskip }
\end{minipage} \\ \cdashline{2-2}

 & Actual Result \\
 & \begin{minipage}[t]{15cm}{\footnotesize

\medskip }
\end{minipage} \\ \cdashline{2-2}

 & Status: \textbf{ Not Executed } \\ \hline

7 & Description \\
 & \begin{minipage}[t]{15cm}
{\footnotesize
Apply appropriate cuts to extract the optimal dataset for comparison,
then calculate statistics (median, 1-sigma range, etc.; also plot a
histogram) of the offsets in milliarcseconds. Confirm that the offset is
less than \textbf{astrometricAccuracy}.

\medskip }
\end{minipage}
\\ \cdashline{2-2}


 & Expected Result \\
 & \begin{minipage}[t]{15cm}{\footnotesize
Histogram and relevant statistics needed to confirm that the WCS
transformation is accurate.

\medskip }
\end{minipage} \\ \cdashline{2-2}

 & Actual Result \\
 & \begin{minipage}[t]{15cm}{\footnotesize

\medskip }
\end{minipage} \\ \cdashline{2-2}

 & Status: \textbf{ Not Executed } \\ \hline

8 & Description \\
 & \begin{minipage}[t]{15cm}
{\footnotesize
Repeat Step 5, but for subregions of the image, to confirm that the
accuracy criterion is met at all positions.

\medskip }
\end{minipage}
\\ \cdashline{2-2}


 & Expected Result \\
 & \begin{minipage}[t]{15cm}{\footnotesize
\textbf{astrometricAccuracy~}requirement is met over the entire image.

\medskip }
\end{minipage} \\ \cdashline{2-2}

 & Actual Result \\
 & \begin{minipage}[t]{15cm}{\footnotesize

\medskip }
\end{minipage} \\ \cdashline{2-2}

 & Status: \textbf{ Not Executed } \\ \hline

\end{longtable}

\paragraph{ LVV-T1759 - Verify calculation of photometric outliers in gri bands }\mbox{}\\

Version \textbf{1}.
Open  \href{https://jira.lsstcorp.org/secure/Tests.jspa#/testCase/LVV-T1759}{\textit{ LVV-T1759 } }
test case in Jira.

Verify that the DM system has provided the code to calculate the
photometric repeatability in the g, r, and i filters, and assess whether
it meets the requirement that no more than \textbf{PF1 =
10{[}percent{]}} of the repeatability outliers exceed the outlier limit
of \textbf{PA2gri = 15 millimagnitudes}.

\textbf{ Preconditions}:\\


Execution status: {\bf Not Executed }

Final comment:\\


Detailed steps results:

\begin{longtable}{p{1cm}p{15cm}}
\hline
{Step} & Step Details\\ \hline
1 & Description \\
 & \begin{minipage}[t]{15cm}
{\footnotesize
Identify a dataset containing at least one field in each of the g, r,
and i filters with multiple overlapping visits.

\medskip }
\end{minipage}
\\ \cdashline{2-2}


 & Expected Result \\
 & \begin{minipage}[t]{15cm}{\footnotesize
A dataset that has been ingested into a Butler repository.

\medskip }
\end{minipage} \\ \cdashline{2-2}

 & Actual Result \\
 & \begin{minipage}[t]{15cm}{\footnotesize

\medskip }
\end{minipage} \\ \cdashline{2-2}

 & Status: \textbf{ Not Executed } \\ \hline

2 & Description \\
 & \begin{minipage}[t]{15cm}
{\footnotesize
The `path` that you will use depends on where you are running the
science pipelines. Options:\\[2\baselineskip]

\begin{itemize}
\tightlist
\item
  local (newinstall.sh - based
  install):{[}path\_to\_installation{]}/loadLSST.bash
\item
  development cluster (``lsst-dev''):
  /software/lsstsw/stack/loadLSST.bash
\item
  LSP Notebook aspect (from a terminal):
  /opt/lsst/software/stack/loadLSST.bash
\end{itemize}

From the command line, execute the commands below in the example
code:\\[2\baselineskip]

\medskip }
\end{minipage}
\\ \cdashline{2-2}

 & Example Code \\
 & \begin{minipage}[t]{15cm}{\footnotesize
source `path`\\
setup lsst\_distrib

\medskip }
\end{minipage} \\ \cdashline{2-2}

 & Expected Result \\
 & \begin{minipage}[t]{15cm}{\footnotesize
Science pipeline software is available for use. If additional packages
are needed (for example, `obs' packages such as `obs\_subaru`), then
additional `setup` commands will be necessary.\\[2\baselineskip]To check
versions in use, type:\\
eups list -s

\medskip }
\end{minipage} \\ \cdashline{2-2}

 & Actual Result \\
 & \begin{minipage}[t]{15cm}{\footnotesize

\medskip }
\end{minipage} \\ \cdashline{2-2}

 & Status: \textbf{ Not Executed } \\ \hline

3 & Description \\
 & \begin{minipage}[t]{15cm}
{\footnotesize
Execute `validate\_drp` on a repository containing precursor data.
Identify the path to the data, which we will call `DATA/path', then
execute the following (with additional flags specified as needed):

\medskip }
\end{minipage}
\\ \cdashline{2-2}

 & Example Code \\
 & \begin{minipage}[t]{15cm}{\footnotesize
validateDrp.py `DATA/path`

\medskip }
\end{minipage} \\ \cdashline{2-2}

 & Expected Result \\
 & \begin{minipage}[t]{15cm}{\footnotesize
JSON files (and associated figures) containing the Measurements and any
associated ``extras.''

\medskip }
\end{minipage} \\ \cdashline{2-2}

 & Actual Result \\
 & \begin{minipage}[t]{15cm}{\footnotesize

\medskip }
\end{minipage} \\ \cdashline{2-2}

 & Status: \textbf{ Not Executed } \\ \hline

4 & Description \\
 & \begin{minipage}[t]{15cm}
{\footnotesize
Confirm that the metric PA2gri has been calculated using the threshold
PF1, and that its values are reasonable.

\medskip }
\end{minipage}
\\ \cdashline{2-2}


 & Expected Result \\
 & \begin{minipage}[t]{15cm}{\footnotesize
A JSON file (and/or a report generated from that JSON file)
demonstrating that PA2gri has been calculated (and that it used PF1).

\medskip }
\end{minipage} \\ \cdashline{2-2}

 & Actual Result \\
 & \begin{minipage}[t]{15cm}{\footnotesize

\medskip }
\end{minipage} \\ \cdashline{2-2}

 & Status: \textbf{ Not Executed } \\ \hline

\end{longtable}

\paragraph{ LVV-T1758 - Verify calculation of photometric outliers in uzy bands }\mbox{}\\

Version \textbf{1}.
Open  \href{https://jira.lsstcorp.org/secure/Tests.jspa#/testCase/LVV-T1758}{\textit{ LVV-T1758 } }
test case in Jira.

Verify that the DM system has provided the code to calculate the
photometric repeatability in the u, z, and y filters, and assess whether
it meets the requirement that no more than \textbf{PF1 =
10{[}percent{]}} of the repeatability outliers exceed the outlier limit
of \textbf{PA2uzy = 22.5 millimagnitudes}.~

\textbf{ Preconditions}:\\


Execution status: {\bf Not Executed }

Final comment:\\


Detailed steps results:

\begin{longtable}{p{1cm}p{15cm}}
\hline
{Step} & Step Details\\ \hline
1 & Description \\
 & \begin{minipage}[t]{15cm}
{\footnotesize
Identify a dataset containing at least one field in each of the u, z,
and y filters with multiple overlapping visits.

\medskip }
\end{minipage}
\\ \cdashline{2-2}


 & Expected Result \\
 & \begin{minipage}[t]{15cm}{\footnotesize
A dataset that has been ingested into a Butler repository.

\medskip }
\end{minipage} \\ \cdashline{2-2}

 & Actual Result \\
 & \begin{minipage}[t]{15cm}{\footnotesize

\medskip }
\end{minipage} \\ \cdashline{2-2}

 & Status: \textbf{ Not Executed } \\ \hline

2 & Description \\
 & \begin{minipage}[t]{15cm}
{\footnotesize
The `path` that you will use depends on where you are running the
science pipelines. Options:\\[2\baselineskip]

\begin{itemize}
\tightlist
\item
  local (newinstall.sh - based
  install):{[}path\_to\_installation{]}/loadLSST.bash
\item
  development cluster (``lsst-dev''):
  /software/lsstsw/stack/loadLSST.bash
\item
  LSP Notebook aspect (from a terminal):
  /opt/lsst/software/stack/loadLSST.bash
\end{itemize}

From the command line, execute the commands below in the example
code:\\[2\baselineskip]

\medskip }
\end{minipage}
\\ \cdashline{2-2}

 & Example Code \\
 & \begin{minipage}[t]{15cm}{\footnotesize
source `path`\\
setup lsst\_distrib

\medskip }
\end{minipage} \\ \cdashline{2-2}

 & Expected Result \\
 & \begin{minipage}[t]{15cm}{\footnotesize
Science pipeline software is available for use. If additional packages
are needed (for example, `obs' packages such as `obs\_subaru`), then
additional `setup` commands will be necessary.\\[2\baselineskip]To check
versions in use, type:\\
eups list -s

\medskip }
\end{minipage} \\ \cdashline{2-2}

 & Actual Result \\
 & \begin{minipage}[t]{15cm}{\footnotesize

\medskip }
\end{minipage} \\ \cdashline{2-2}

 & Status: \textbf{ Not Executed } \\ \hline

3 & Description \\
 & \begin{minipage}[t]{15cm}
{\footnotesize
Execute `validate\_drp` on a repository containing precursor data.
Identify the path to the data, which we will call `DATA/path', then
execute the following (with additional flags specified as needed):

\medskip }
\end{minipage}
\\ \cdashline{2-2}

 & Example Code \\
 & \begin{minipage}[t]{15cm}{\footnotesize
validateDrp.py `DATA/path`

\medskip }
\end{minipage} \\ \cdashline{2-2}

 & Expected Result \\
 & \begin{minipage}[t]{15cm}{\footnotesize
JSON files (and associated figures) containing the Measurements and any
associated ``extras.''

\medskip }
\end{minipage} \\ \cdashline{2-2}

 & Actual Result \\
 & \begin{minipage}[t]{15cm}{\footnotesize

\medskip }
\end{minipage} \\ \cdashline{2-2}

 & Status: \textbf{ Not Executed } \\ \hline

4 & Description \\
 & \begin{minipage}[t]{15cm}
{\footnotesize
Confirm that the metric PA2uzy has been calculated using the threshold
PF1, and that its values are reasonable.

\medskip }
\end{minipage}
\\ \cdashline{2-2}


 & Expected Result \\
 & \begin{minipage}[t]{15cm}{\footnotesize
A JSON file (and/or a report generated from that JSON file)
demonstrating that PA2uzy has been calculated (and that it used PF1).

\medskip }
\end{minipage} \\ \cdashline{2-2}

 & Actual Result \\
 & \begin{minipage}[t]{15cm}{\footnotesize

\medskip }
\end{minipage} \\ \cdashline{2-2}

 & Status: \textbf{ Not Executed } \\ \hline

\end{longtable}

\paragraph{ LVV-T1756 - Verify calculation of photometric repeatability in uzy filters }\mbox{}\\

Version \textbf{1}.
Open  \href{https://jira.lsstcorp.org/secure/Tests.jspa#/testCase/LVV-T1756}{\textit{ LVV-T1756 } }
test case in Jira.

Verify that the DM system has provided the code to calculate the RMS
photometric repeatability of bright non-saturated unresolved point
sources in the u, z, and y filters, and assess whether it meets the
requirement that it shall be less than \textbf{PA1uzy = 7.5
millimagnitudes}.

\textbf{ Preconditions}:\\


Execution status: {\bf Not Executed }

Final comment:\\


Detailed steps results:

\begin{longtable}{p{1cm}p{15cm}}
\hline
{Step} & Step Details\\ \hline
1 & Description \\
 & \begin{minipage}[t]{15cm}
{\footnotesize
Identify a dataset containing at least one field in each of the u, z,
and y filters with multiple overlapping visits.

\medskip }
\end{minipage}
\\ \cdashline{2-2}


 & Expected Result \\
 & \begin{minipage}[t]{15cm}{\footnotesize
A dataset that has been ingested into a Butler repository.

\medskip }
\end{minipage} \\ \cdashline{2-2}

 & Actual Result \\
 & \begin{minipage}[t]{15cm}{\footnotesize

\medskip }
\end{minipage} \\ \cdashline{2-2}

 & Status: \textbf{ Not Executed } \\ \hline

2 & Description \\
 & \begin{minipage}[t]{15cm}
{\footnotesize
Execute `validate\_drp` on a repository containing precursor data.
Identify the path to the data, which we will call `DATA/path', then
execute the following (with additional flags specified as needed):

\medskip }
\end{minipage}
\\ \cdashline{2-2}

 & Example Code \\
 & \begin{minipage}[t]{15cm}{\footnotesize
validateDrp.py `DATA/path`

\medskip }
\end{minipage} \\ \cdashline{2-2}

 & Expected Result \\
 & \begin{minipage}[t]{15cm}{\footnotesize
JSON files (and associated figures) containing the Measurements and any
associated ``extras.''

\medskip }
\end{minipage} \\ \cdashline{2-2}

 & Actual Result \\
 & \begin{minipage}[t]{15cm}{\footnotesize

\medskip }
\end{minipage} \\ \cdashline{2-2}

 & Status: \textbf{ Not Executed } \\ \hline

3 & Description \\
 & \begin{minipage}[t]{15cm}
{\footnotesize
Confirm that the metric PA1uzy has been calculated, and that its values
are reasonable.

\medskip }
\end{minipage}
\\ \cdashline{2-2}


 & Expected Result \\
 & \begin{minipage}[t]{15cm}{\footnotesize
A JSON file (and/or a report generated from that JSON file)
demonstrating that PA1uzy has been calculated.

\medskip }
\end{minipage} \\ \cdashline{2-2}

 & Actual Result \\
 & \begin{minipage}[t]{15cm}{\footnotesize

\medskip }
\end{minipage} \\ \cdashline{2-2}

 & Status: \textbf{ Not Executed } \\ \hline

\end{longtable}

\paragraph{ LVV-T1757 - Verify calculation of photometric repeatability in gri filters }\mbox{}\\

Version \textbf{1}.
Open  \href{https://jira.lsstcorp.org/secure/Tests.jspa#/testCase/LVV-T1757}{\textit{ LVV-T1757 } }
test case in Jira.

Verify that the DM system has provided the code to calculate the RMS
photometric repeatability of bright non-saturated unresolved point
sources in the g, r, and i filters, and assess whether it meets the
requirement that it shall be less than \textbf{PA1gri = 5.0
millimagnitudes}.

\textbf{ Preconditions}:\\


Execution status: {\bf Not Executed }

Final comment:\\


Detailed steps results:

\begin{longtable}{p{1cm}p{15cm}}
\hline
{Step} & Step Details\\ \hline
1 & Description \\
 & \begin{minipage}[t]{15cm}
{\footnotesize
Identify a dataset containing at least one field in each of the g, r,
and i filters with multiple overlapping visits.

\medskip }
\end{minipage}
\\ \cdashline{2-2}


 & Expected Result \\
 & \begin{minipage}[t]{15cm}{\footnotesize
A dataset that has been ingested into a Butler repository.

\medskip }
\end{minipage} \\ \cdashline{2-2}

 & Actual Result \\
 & \begin{minipage}[t]{15cm}{\footnotesize

\medskip }
\end{minipage} \\ \cdashline{2-2}

 & Status: \textbf{ Not Executed } \\ \hline

2 & Description \\
 & \begin{minipage}[t]{15cm}
{\footnotesize
Execute `validate\_drp` on a repository containing precursor data.
Identify the path to the data, which we will call `DATA/path', then
execute the following (with additional flags specified as needed):

\medskip }
\end{minipage}
\\ \cdashline{2-2}

 & Example Code \\
 & \begin{minipage}[t]{15cm}{\footnotesize
validateDrp.py `DATA/path`

\medskip }
\end{minipage} \\ \cdashline{2-2}

 & Expected Result \\
 & \begin{minipage}[t]{15cm}{\footnotesize
JSON files (and associated figures) containing the Measurements and any
associated ``extras.''

\medskip }
\end{minipage} \\ \cdashline{2-2}

 & Actual Result \\
 & \begin{minipage}[t]{15cm}{\footnotesize

\medskip }
\end{minipage} \\ \cdashline{2-2}

 & Status: \textbf{ Not Executed } \\ \hline

3 & Description \\
 & \begin{minipage}[t]{15cm}
{\footnotesize
Confirm that the metric PA1gri has been calculated, and that its values
are reasonable.

\medskip }
\end{minipage}
\\ \cdashline{2-2}


 & Expected Result \\
 & \begin{minipage}[t]{15cm}{\footnotesize
A JSON file (and/or a report generated from that JSON file)
demonstrating that PA1gri has been calculated.

\medskip }
\end{minipage} \\ \cdashline{2-2}

 & Actual Result \\
 & \begin{minipage}[t]{15cm}{\footnotesize

\medskip }
\end{minipage} \\ \cdashline{2-2}

 & Status: \textbf{ Not Executed } \\ \hline

\end{longtable}

\paragraph{ LVV-T125 - Verify implementation of Simulated Data }\mbox{}\\

Version \textbf{1}.
Open  \href{https://jira.lsstcorp.org/secure/Tests.jspa#/testCase/LVV-T125}{\textit{ LVV-T125 } }
test case in Jira.

Verify that the DMS can inject simulated data into data products for
testing.

\textbf{ Preconditions}:\\


Execution status: {\bf Not Executed }

Final comment:\\


Detailed steps results:

\begin{longtable}{p{1cm}p{15cm}}
\hline
{Step} & Step Details\\ \hline
1 & Description \\
 & \begin{minipage}[t]{15cm}
{\footnotesize
Identify a dataset that has been (or can be readily) processed through
single-frame processing and coaddition.

\medskip }
\end{minipage}
\\ \cdashline{2-2}


 & Expected Result \\
 & \begin{minipage}[t]{15cm}{\footnotesize
The `calexp` and `deepCoadd\_calexp` images and their associated source
catalogs are created.

\medskip }
\end{minipage} \\ \cdashline{2-2}

 & Actual Result \\
 & \begin{minipage}[t]{15cm}{\footnotesize

\medskip }
\end{minipage} \\ \cdashline{2-2}

 & Status: \textbf{ Not Executed } \\ \hline

2 & Description \\
 & \begin{minipage}[t]{15cm}
{\footnotesize
Roughly determine the coordinates of a bounding box that is contained
within the images that were processed.

\medskip }
\end{minipage}
\\ \cdashline{2-2}


 & Expected Result \\
 & \begin{minipage}[t]{15cm}{\footnotesize
RA, Dec boundaries of a region in which to generate fake sources.

\medskip }
\end{minipage} \\ \cdashline{2-2}

 & Actual Result \\
 & \begin{minipage}[t]{15cm}{\footnotesize

\medskip }
\end{minipage} \\ \cdashline{2-2}

 & Status: \textbf{ Not Executed } \\ \hline

3 & Description \\
 & \begin{minipage}[t]{15cm}
{\footnotesize
Generate a catalog in the correct format for `insertFakes` to accept.
The catalog should specify positions and magnitudes of stars (and
optionally, parameters specifying galaxy shape, if galaxies are also
being inserted).

\medskip }
\end{minipage}
\\ \cdashline{2-2}


 & Expected Result \\
 & \begin{minipage}[t]{15cm}{\footnotesize
An input catalog of fake source positions and magnitudes to be inserted
into the images.

\medskip }
\end{minipage} \\ \cdashline{2-2}

 & Actual Result \\
 & \begin{minipage}[t]{15cm}{\footnotesize

\medskip }
\end{minipage} \\ \cdashline{2-2}

 & Status: \textbf{ Not Executed } \\ \hline

4 & Description \\
 & \begin{minipage}[t]{15cm}
{\footnotesize
Execute `insertFakes.py` on the repository, specifying the input catalog
from the previous step.

\medskip }
\end{minipage}
\\ \cdashline{2-2}


 & Expected Result \\
 & \begin{minipage}[t]{15cm}{\footnotesize
A repository with images that have fake sources inserted.

\medskip }
\end{minipage} \\ \cdashline{2-2}

 & Actual Result \\
 & \begin{minipage}[t]{15cm}{\footnotesize

\medskip }
\end{minipage} \\ \cdashline{2-2}

 & Status: \textbf{ Not Executed } \\ \hline

5 & Description \\
 & \begin{minipage}[t]{15cm}
{\footnotesize
Run `multiBandDriver.py` on the repository, specifying the fake-source
repository as the input.

\medskip }
\end{minipage}
\\ \cdashline{2-2}


 & Expected Result \\
 & \begin{minipage}[t]{15cm}{\footnotesize
`calexp` and coadd images containing the artificial sources and sources
catalogs that contain their measurements along with the sources detected
in the original run.

\medskip }
\end{minipage} \\ \cdashline{2-2}

 & Actual Result \\
 & \begin{minipage}[t]{15cm}{\footnotesize

\medskip }
\end{minipage} \\ \cdashline{2-2}

 & Status: \textbf{ Not Executed } \\ \hline

6 & Description \\
 & \begin{minipage}[t]{15cm}
{\footnotesize
Confirm that the injected sources appear in the images and the catalogs.

\medskip }
\end{minipage}
\\ \cdashline{2-2}


 & Expected Result \\
 & \begin{minipage}[t]{15cm}{\footnotesize
Fake sources and their measured properties are recoverable.

\medskip }
\end{minipage} \\ \cdashline{2-2}

 & Actual Result \\
 & \begin{minipage}[t]{15cm}{\footnotesize

\medskip }
\end{minipage} \\ \cdashline{2-2}

 & Status: \textbf{ Not Executed } \\ \hline

\end{longtable}

\paragraph{ LVV-T36 - Verify implementation of Difference Exposures }\mbox{}\\

Version \textbf{1}.
Open  \href{https://jira.lsstcorp.org/secure/Tests.jspa#/testCase/LVV-T36}{\textit{ LVV-T36 } }
test case in Jira.

Verify successful creation of a\\
1. PSF-matched template image for a given Processed Visit Image\\
2. Difference Exposure from each Processed Visit Image

\textbf{ Preconditions}:\\


Execution status: {\bf Not Executed }

Final comment:\\


Detailed steps results:

\begin{longtable}{p{1cm}p{15cm}}
\hline
{Step} & Step Details\\ \hline
1 & Description \\
 & \begin{minipage}[t]{15cm}
{\footnotesize
The `path` that you will use depends on where you are running the
science pipelines. Options:\\[2\baselineskip]

\begin{itemize}
\tightlist
\item
  local (newinstall.sh - based
  install):{[}path\_to\_installation{]}/loadLSST.bash
\item
  development cluster (``lsst-dev''):
  /software/lsstsw/stack/loadLSST.bash
\item
  LSP Notebook aspect (from a terminal):
  /opt/lsst/software/stack/loadLSST.bash
\end{itemize}

From the command line, execute the commands below in the example
code:\\[2\baselineskip]

\medskip }
\end{minipage}
\\ \cdashline{2-2}

 & Example Code \\
 & \begin{minipage}[t]{15cm}{\footnotesize
source `path`\\
setup lsst\_distrib

\medskip }
\end{minipage} \\ \cdashline{2-2}

 & Expected Result \\
 & \begin{minipage}[t]{15cm}{\footnotesize
Science pipeline software is available for use. If additional packages
are needed (for example, `obs' packages such as `obs\_subaru`), then
additional `setup` commands will be necessary.\\[2\baselineskip]To check
versions in use, type:\\
eups list -s

\medskip }
\end{minipage} \\ \cdashline{2-2}

 & Actual Result \\
 & \begin{minipage}[t]{15cm}{\footnotesize

\medskip }
\end{minipage} \\ \cdashline{2-2}

 & Status: \textbf{ Not Executed } \\ \hline

2 & Description \\
 & \begin{minipage}[t]{15cm}
{\footnotesize
Perform the steps of Alert Production (including, but not necessarily
limited to, single frame processing, ISR, source detection/measurement,
PSF estimation, photometric and astrometric calibration, difference
imaging, DIASource detection/measurement, source association). During
Operations, it is presumed that these are automated for a given
dataset.~

\medskip }
\end{minipage}
\\ \cdashline{2-2}


 & Expected Result \\
 & \begin{minipage}[t]{15cm}{\footnotesize
An output dataset including difference images and DIASource and
DIAObject measurements.

\medskip }
\end{minipage} \\ \cdashline{2-2}

 & Actual Result \\
 & \begin{minipage}[t]{15cm}{\footnotesize

\medskip }
\end{minipage} \\ \cdashline{2-2}

 & Status: \textbf{ Not Executed } \\ \hline

3 & Description \\
 & \begin{minipage}[t]{15cm}
{\footnotesize
Verify that the expected data products have been produced, and that
catalogs contain reasonable values for measured quantities of interest.

\medskip }
\end{minipage}
\\ \cdashline{2-2}


 & Expected Result \\
 & \begin{minipage}[t]{15cm}{\footnotesize

\medskip }
\end{minipage} \\ \cdashline{2-2}

 & Actual Result \\
 & \begin{minipage}[t]{15cm}{\footnotesize

\medskip }
\end{minipage} \\ \cdashline{2-2}

 & Status: \textbf{ Not Executed } \\ \hline

4 & Description \\
 & \begin{minipage}[t]{15cm}
{\footnotesize
Demonstrate successful creation of a template image from HSC PDF and
DECAM HiTS data. ~Demonstrate successful creation of a Difference
Exposure for at least 10 other images from survey, ideally at a range of
arimass. ~In particular, HiTS has 2013A u-band data. ~While the Blanco
4-m does have an ADC, there are still some chromatic effects and we
should demonstrate that we can successfully produce Difference Exposures
and templates for diferent airmass bins.

\medskip }
\end{minipage}
\\ \cdashline{2-2}


 & Expected Result \\
 & \begin{minipage}[t]{15cm}{\footnotesize

\medskip }
\end{minipage} \\ \cdashline{2-2}

 & Actual Result \\
 & \begin{minipage}[t]{15cm}{\footnotesize

\medskip }
\end{minipage} \\ \cdashline{2-2}

 & Status: \textbf{ Not Executed } \\ \hline

\end{longtable}

\paragraph{ LVV-T126 - Verify implementation of Image Differencing }\mbox{}\\

Version \textbf{1}.
Open  \href{https://jira.lsstcorp.org/secure/Tests.jspa#/testCase/LVV-T126}{\textit{ LVV-T126 } }
test case in Jira.

Verify that the DMS can performance image differencing from single
exposures and coadds.

\textbf{ Preconditions}:\\


Execution status: {\bf Not Executed }

Final comment:\\


Detailed steps results:

\begin{longtable}{p{1cm}p{15cm}}
\hline
{Step} & Step Details\\ \hline
1 & Description \\
 & \begin{minipage}[t]{15cm}
{\footnotesize
Identify a repository containing data that have been processed through
the difference imaging pipeline. (e.g., the HiTS 2015 data that are
processed monthly for testing)

\medskip }
\end{minipage}
\\ \cdashline{2-2}


 & Expected Result \\
 & \begin{minipage}[t]{15cm}{\footnotesize
A dataset containing calexps, difference images, and source catalogs (of
diaSrcs).

\medskip }
\end{minipage} \\ \cdashline{2-2}

 & Actual Result \\
 & \begin{minipage}[t]{15cm}{\footnotesize

\medskip }
\end{minipage} \\ \cdashline{2-2}

 & Status: \textbf{ Not Executed } \\ \hline

2 & Description \\
 & \begin{minipage}[t]{15cm}
{\footnotesize
Identify the path to the data repository, which we will refer to as
`DATA/path', then execute the following:

\medskip }
\end{minipage}
\\ \cdashline{2-2}

 & Example Code \\
 & \begin{minipage}[t]{15cm}{\footnotesize
\begin{verbatim}
import lsst.daf.persistence as dafPersist
butler = dafPersist.Butler(inputs='DATA/path')
\end{verbatim}

\medskip }
\end{minipage} \\ \cdashline{2-2}

 & Expected Result \\
 & \begin{minipage}[t]{15cm}{\footnotesize
Butler repo available for reading.

\medskip }
\end{minipage} \\ \cdashline{2-2}

 & Actual Result \\
 & \begin{minipage}[t]{15cm}{\footnotesize

\medskip }
\end{minipage} \\ \cdashline{2-2}

 & Status: \textbf{ Not Executed } \\ \hline

3 & Description \\
 & \begin{minipage}[t]{15cm}
{\footnotesize
Extract a `calexp`, a `deepDiff\_differenceExp`, and the
`deepDiff\_diaSrc` catalog of measurements.

\medskip }
\end{minipage}
\\ \cdashline{2-2}


 & Expected Result \\
 & \begin{minipage}[t]{15cm}{\footnotesize
Well-formed images and catalogs containing the calexp from the visit
image and the difference image, and measurements of sources from the
difference image.

\medskip }
\end{minipage} \\ \cdashline{2-2}

 & Actual Result \\
 & \begin{minipage}[t]{15cm}{\footnotesize

\medskip }
\end{minipage} \\ \cdashline{2-2}

 & Status: \textbf{ Not Executed } \\ \hline

4 & Description \\
 & \begin{minipage}[t]{15cm}
{\footnotesize
Confirm (by visual inspection) that the difference image is mostly blank
sky (i.e., has had a template of the same field subtracted), and that
the source catalog contains sources with photometric and astrometric
measurements.

\medskip }
\end{minipage}
\\ \cdashline{2-2}


 & Expected Result \\
 & \begin{minipage}[t]{15cm}{\footnotesize
A mostly blank image (with perhaps some artifacts due to imperfect
subtraction) and a catalog of sources detected/measured from that image.

\medskip }
\end{minipage} \\ \cdashline{2-2}

 & Actual Result \\
 & \begin{minipage}[t]{15cm}{\footnotesize

\medskip }
\end{minipage} \\ \cdashline{2-2}

 & Status: \textbf{ Not Executed } \\ \hline

\end{longtable}

\paragraph{ LVV-T39 - Verify implementation of Generate Photometric Zeropoint for Visit Image }\mbox{}\\

Version \textbf{1}.
Open  \href{https://jira.lsstcorp.org/secure/Tests.jspa#/testCase/LVV-T39}{\textit{ LVV-T39 } }
test case in Jira.

Verify that Processed Visit Image data products produced by the DRP and
AP pipelines include the parameters of a model that relates the observed
flux on the image to physical flux units.

\textbf{ Preconditions}:\\


Execution status: {\bf Not Executed }

Final comment:\\


Detailed steps results:

\begin{longtable}{p{1cm}p{15cm}}
\hline
{Step} & Step Details\\ \hline
1 & Description \\
 & \begin{minipage}[t]{15cm}
{\footnotesize
Identify a dataset with processed visit images in multiple filters.

\medskip }
\end{minipage}
\\ \cdashline{2-2}


 & Expected Result \\
 & \begin{minipage}[t]{15cm}{\footnotesize

\medskip }
\end{minipage} \\ \cdashline{2-2}

 & Actual Result \\
 & \begin{minipage}[t]{15cm}{\footnotesize

\medskip }
\end{minipage} \\ \cdashline{2-2}

 & Status: \textbf{ Not Executed } \\ \hline

2 & Description \\
 & \begin{minipage}[t]{15cm}
{\footnotesize
Identify the path to the data repository, which we will refer to as
`DATA/path', then execute the following:

\medskip }
\end{minipage}
\\ \cdashline{2-2}

 & Example Code \\
 & \begin{minipage}[t]{15cm}{\footnotesize
\begin{verbatim}
import lsst.daf.persistence as dafPersist
butler = dafPersist.Butler(inputs='DATA/path')
\end{verbatim}

\medskip }
\end{minipage} \\ \cdashline{2-2}

 & Expected Result \\
 & \begin{minipage}[t]{15cm}{\footnotesize
Butler repo available for reading.

\medskip }
\end{minipage} \\ \cdashline{2-2}

 & Actual Result \\
 & \begin{minipage}[t]{15cm}{\footnotesize

\medskip }
\end{minipage} \\ \cdashline{2-2}

 & Status: \textbf{ Not Executed } \\ \hline

3 & Description \\
 & \begin{minipage}[t]{15cm}
{\footnotesize
Extract the photometric zeropoint from the source catalog associated
with a visit image. Repeat this for all available filters, and confirm
that the zeropoint has been set, and has a reasonable value.

\medskip }
\end{minipage}
\\ \cdashline{2-2}


 & Expected Result \\
 & \begin{minipage}[t]{15cm}{\footnotesize
A zeropoint that enables one to convert the measured fluxes to
magnitudes.

\medskip }
\end{minipage} \\ \cdashline{2-2}

 & Actual Result \\
 & \begin{minipage}[t]{15cm}{\footnotesize

\medskip }
\end{minipage} \\ \cdashline{2-2}

 & Status: \textbf{ Not Executed } \\ \hline

4 & Description \\
 & \begin{minipage}[t]{15cm}
{\footnotesize
Extract fluxes for some sources, and convert them to magnitudes. Confirm
that the distribution spans a reasonable range.

\medskip }
\end{minipage}
\\ \cdashline{2-2}


 & Expected Result \\
 & \begin{minipage}[t]{15cm}{\footnotesize
In most cases, well-measured magnitudes (i.e., for high S/N
measurements) should be between 12 to 28 for all bands.

\medskip }
\end{minipage} \\ \cdashline{2-2}

 & Actual Result \\
 & \begin{minipage}[t]{15cm}{\footnotesize

\medskip }
\end{minipage} \\ \cdashline{2-2}

 & Status: \textbf{ Not Executed } \\ \hline

\end{longtable}

\paragraph{ LVV-T46 - Verify implementation of Prompt Processing Performance Report Definition }\mbox{}\\

Version \textbf{1}.
Open  \href{https://jira.lsstcorp.org/secure/Tests.jspa#/testCase/LVV-T46}{\textit{ LVV-T46 } }
test case in Jira.

Verify that the DMS produces a Prompt Processing Performance Report.
~Specifically check that the number of observations that describe each
of the following:\\
1. Successfully processed, recoverable failures, unrecoverable
failures.\\
2. Archived\\
3. Result in science.\\[2\baselineskip]This is testing more the
processing rather than the observatory system.

\textbf{ Preconditions}:\\


Execution status: {\bf Not Executed }

Final comment:\\


Detailed steps results:

\begin{longtable}{p{1cm}p{15cm}}
\hline
{Step} & Step Details\\ \hline
1 & Description \\
 & \begin{minipage}[t]{15cm}
{\footnotesize
Execute single-day operations rehearsal, observe report

\medskip }
\end{minipage}
\\ \cdashline{2-2}


 & Expected Result \\
 & \begin{minipage}[t]{15cm}{\footnotesize

\medskip }
\end{minipage} \\ \cdashline{2-2}

 & Actual Result \\
 & \begin{minipage}[t]{15cm}{\footnotesize

\medskip }
\end{minipage} \\ \cdashline{2-2}

 & Status: \textbf{ Not Executed } \\ \hline

\end{longtable}

\paragraph{ LVV-T38 - Verify implementation of Processed Visit Images }\mbox{}\\

Version \textbf{1}.
Open  \href{https://jira.lsstcorp.org/secure/Tests.jspa#/testCase/LVV-T38}{\textit{ LVV-T38 } }
test case in Jira.

Verify that the DMS\\
1. Successfully produces Processed Visit Images, where the instrument
signature has been removed.\\
2. Successfully combines images obtained during a standard visit.

\textbf{ Preconditions}:\\


Execution status: {\bf Not Executed }

Final comment:\\


Detailed steps results:

\begin{longtable}{p{1cm}p{15cm}}
\hline
{Step} & Step Details\\ \hline
1 & Description \\
 & \begin{minipage}[t]{15cm}
{\footnotesize
Identify suitable precursor datasets containing unprocessed raw images.

\medskip }
\end{minipage}
\\ \cdashline{2-2}


 & Expected Result \\
 & \begin{minipage}[t]{15cm}{\footnotesize

\medskip }
\end{minipage} \\ \cdashline{2-2}

 & Actual Result \\
 & \begin{minipage}[t]{15cm}{\footnotesize

\medskip }
\end{minipage} \\ \cdashline{2-2}

 & Status: \textbf{ Not Executed } \\ \hline

2 & Description \\
 & \begin{minipage}[t]{15cm}
{\footnotesize
Run the Prompt Processing payload on these data. ~Verify that Processed
Visit Images are generated at correct size and with significant
instrumental artifacts removed.

\medskip }
\end{minipage}
\\ \cdashline{2-2}


 & Expected Result \\
 & \begin{minipage}[t]{15cm}{\footnotesize
Raw precursor dataset images have been processed into Processed Visit
Images, with instrumental artifacts corrected.

\medskip }
\end{minipage} \\ \cdashline{2-2}

 & Actual Result \\
 & \begin{minipage}[t]{15cm}{\footnotesize

\medskip }
\end{minipage} \\ \cdashline{2-2}

 & Status: \textbf{ Not Executed } \\ \hline

3 & Description \\
 & \begin{minipage}[t]{15cm}
{\footnotesize
Run camera test stand data through full acquisition+backbone+ISR.

\medskip }
\end{minipage}
\\ \cdashline{2-2}


 & Expected Result \\
 & \begin{minipage}[t]{15cm}{\footnotesize

\medskip }
\end{minipage} \\ \cdashline{2-2}

 & Actual Result \\
 & \begin{minipage}[t]{15cm}{\footnotesize

\medskip }
\end{minipage} \\ \cdashline{2-2}

 & Status: \textbf{ Not Executed } \\ \hline

4 & Description \\
 & \begin{minipage}[t]{15cm}
{\footnotesize
Run simulated LSST data with calibrations through prompt processing
system and inspect Processed Visit images to verify that they have been
cleaned of significant artifacts and are of the correct, shape, and
described orientation.

\medskip }
\end{minipage}
\\ \cdashline{2-2}


 & Expected Result \\
 & \begin{minipage}[t]{15cm}{\footnotesize
Raw images have been processed into Processed Visit Images, with
instrumental artifacts corrected.

\medskip }
\end{minipage} \\ \cdashline{2-2}

 & Actual Result \\
 & \begin{minipage}[t]{15cm}{\footnotesize

\medskip }
\end{minipage} \\ \cdashline{2-2}

 & Status: \textbf{ Not Executed } \\ \hline

\end{longtable}

\paragraph{ LVV-T42 - Verify implementation of Processed Visit Image Content }\mbox{}\\

Version \textbf{1}.
Open  \href{https://jira.lsstcorp.org/secure/Tests.jspa#/testCase/LVV-T42}{\textit{ LVV-T42 } }
test case in Jira.

Verify that Processed Visit Images produced by the DRP and AP pipelines
include the observed data, a mask array, a variance array, a PSF model,
and a WCS model.

\textbf{ Preconditions}:\\


Execution status: {\bf Not Executed }

Final comment:\\


Detailed steps results:

\begin{longtable}{p{1cm}p{15cm}}
\hline
{Step} & Step Details\\ \hline
1 & Description \\
 & \begin{minipage}[t]{15cm}
{\footnotesize
Identify the path to the data repository, which we will refer to as
`DATA/path', then execute the following:

\medskip }
\end{minipage}
\\ \cdashline{2-2}

 & Example Code \\
 & \begin{minipage}[t]{15cm}{\footnotesize
\begin{verbatim}
import lsst.daf.persistence as dafPersist
butler = dafPersist.Butler(inputs='DATA/path')
\end{verbatim}

\medskip }
\end{minipage} \\ \cdashline{2-2}

 & Expected Result \\
 & \begin{minipage}[t]{15cm}{\footnotesize
Butler repo available for reading.

\medskip }
\end{minipage} \\ \cdashline{2-2}

 & Actual Result \\
 & \begin{minipage}[t]{15cm}{\footnotesize

\medskip }
\end{minipage} \\ \cdashline{2-2}

 & Status: \textbf{ Not Executed } \\ \hline

2 & Description \\
 & \begin{minipage}[t]{15cm}
{\footnotesize
Ingest the data from an appropriate processed dataset.

\medskip }
\end{minipage}
\\ \cdashline{2-2}


 & Expected Result \\
 & \begin{minipage}[t]{15cm}{\footnotesize

\medskip }
\end{minipage} \\ \cdashline{2-2}

 & Actual Result \\
 & \begin{minipage}[t]{15cm}{\footnotesize

\medskip }
\end{minipage} \\ \cdashline{2-2}

 & Status: \textbf{ Not Executed } \\ \hline

3 & Description \\
 & \begin{minipage}[t]{15cm}
{\footnotesize
Select a single visit from the dataset, and extract its WCS object,
calexp image, psf model, and source list.

\medskip }
\end{minipage}
\\ \cdashline{2-2}


 & Expected Result \\
 & \begin{minipage}[t]{15cm}{\footnotesize

\medskip }
\end{minipage} \\ \cdashline{2-2}

 & Actual Result \\
 & \begin{minipage}[t]{15cm}{\footnotesize

\medskip }
\end{minipage} \\ \cdashline{2-2}

 & Status: \textbf{ Not Executed } \\ \hline

4 & Description \\
 & \begin{minipage}[t]{15cm}
{\footnotesize
Inspect the calexp image to ensure that

\begin{enumerate}
\tightlist
\item
  A well-formed image is present,
\item
  The variance plane is present and well-behaved,
\item
  Mask planes are present and contain information about defects.
\end{enumerate}

\medskip }
\end{minipage}
\\ \cdashline{2-2}


 & Expected Result \\
 & \begin{minipage}[t]{15cm}{\footnotesize
An astronomical image with mask and variance planes. This can be readily
visualized using Firefly, which displays mask planes by default.

\medskip }
\end{minipage} \\ \cdashline{2-2}

 & Actual Result \\
 & \begin{minipage}[t]{15cm}{\footnotesize

\medskip }
\end{minipage} \\ \cdashline{2-2}

 & Status: \textbf{ Not Executed } \\ \hline

5 & Description \\
 & \begin{minipage}[t]{15cm}
{\footnotesize
Plot images of the PSF model at various points, and verify that the PSF
differs with position.

\medskip }
\end{minipage}
\\ \cdashline{2-2}


 & Expected Result \\
 & \begin{minipage}[t]{15cm}{\footnotesize
A ``star-like'' image of the PSF evaluated at various positions. The PSF
should vary slightly with position (this could be readily visualized by
taking a difference of PSFs at two positions).

\medskip }
\end{minipage} \\ \cdashline{2-2}

 & Actual Result \\
 & \begin{minipage}[t]{15cm}{\footnotesize

\medskip }
\end{minipage} \\ \cdashline{2-2}

 & Status: \textbf{ Not Executed } \\ \hline

6 & Description \\
 & \begin{minipage}[t]{15cm}
{\footnotesize
Starting from the XY pixel coordinates of the sources, apply the WCS to
obtain RA, Dec coordinates. Plot these positions and confirm that they
match the expected values from the WCS object.

\medskip }
\end{minipage}
\\ \cdashline{2-2}


 & Expected Result \\
 & \begin{minipage}[t]{15cm}{\footnotesize
RA, Dec coordinates that are returned should be near the central
position of the visit coordinate as given in either the calexp metadata
or the WCS.

\medskip }
\end{minipage} \\ \cdashline{2-2}

 & Actual Result \\
 & \begin{minipage}[t]{15cm}{\footnotesize

\medskip }
\end{minipage} \\ \cdashline{2-2}

 & Status: \textbf{ Not Executed } \\ \hline

7 & Description \\
 & \begin{minipage}[t]{15cm}
{\footnotesize
Repeat steps 2-6, but now with difference images created by the Alert
Production pipeline (for example, in the `ap\_verify` test data
processing).

\medskip }
\end{minipage}
\\ \cdashline{2-2}


 & Expected Result \\
 & \begin{minipage}[t]{15cm}{\footnotesize

\medskip }
\end{minipage} \\ \cdashline{2-2}

 & Actual Result \\
 & \begin{minipage}[t]{15cm}{\footnotesize

\medskip }
\end{minipage} \\ \cdashline{2-2}

 & Status: \textbf{ Not Executed } \\ \hline

\end{longtable}

\paragraph{ LVV-T149 - Verify implementation of Catalog Queries }\mbox{}\\

Version \textbf{1}.
Open  \href{https://jira.lsstcorp.org/secure/Tests.jspa#/testCase/LVV-T149}{\textit{ LVV-T149 } }
test case in Jira.

Verify that SQL, or a similar structured language, can be used to query
catalogs.

\textbf{ Preconditions}:\\
An operational QSERV database that has been verified via
\href{https://jira.lsstcorp.org/secure/Tests.jspa\#/testCase/LVV-T1085}{LVV-T1085}
and
\href{https://jira.lsstcorp.org/secure/Tests.jspa\#/testCase/LVV-T1086}{LVV-T1086}
and
\href{https://jira.lsstcorp.org/secure/Tests.jspa\#/testCase/LVV-T1087}{LVV-T1087}.

Execution status: {\bf Not Executed }

Final comment:\\


Detailed steps results:

\begin{longtable}{p{1cm}p{15cm}}
\hline
{Step} & Step Details\\ \hline
1 & Description \\
 & \begin{minipage}[t]{15cm}
{\footnotesize
Execute a simple query (for example, the one below) and confirm that it
returns the expected result.

\medskip }
\end{minipage}
\\ \cdashline{2-2}

 & Example Code \\
 & \begin{minipage}[t]{15cm}{\footnotesize
SELECT * FROM Object WHERE qserv\_areaspec\_box(316.582327, −6.839078,
316.653938, −6.781822)

\medskip }
\end{minipage} \\ \cdashline{2-2}

 & Expected Result \\
 & \begin{minipage}[t]{15cm}{\footnotesize
A catalog of objects satisfying the specified constraints.~

\medskip }
\end{minipage} \\ \cdashline{2-2}

 & Actual Result \\
 & \begin{minipage}[t]{15cm}{\footnotesize

\medskip }
\end{minipage} \\ \cdashline{2-2}

 & Status: \textbf{ Not Executed } \\ \hline

2 & Description \\
 & \begin{minipage}[t]{15cm}
{\footnotesize
Repeat the query from all available access routes (e.g., an external VO
client, internal DM tools on the development cluster, the Science
Platform query tool, and from within the Notebook Aspect), confirming in
each case that the results are as expected.

\medskip }
\end{minipage}
\\ \cdashline{2-2}


 & Expected Result \\
 & \begin{minipage}[t]{15cm}{\footnotesize

\medskip }
\end{minipage} \\ \cdashline{2-2}

 & Actual Result \\
 & \begin{minipage}[t]{15cm}{\footnotesize

\medskip }
\end{minipage} \\ \cdashline{2-2}

 & Status: \textbf{ Not Executed } \\ \hline

\end{longtable}

\paragraph{ LVV-T151 - Verify Implementation of Catalog Export Formats From the Notebook Aspect }\mbox{}\\

Version \textbf{1}.
Open  \href{https://jira.lsstcorp.org/secure/Tests.jspa#/testCase/LVV-T151}{\textit{ LVV-T151 } }
test case in Jira.

Verify that catalog data is exportable from the notebook aspect in a
variety of community-standard formats.

\textbf{ Preconditions}:\\


Execution status: {\bf Not Executed }

Final comment:\\


Detailed steps results:

\begin{longtable}{p{1cm}p{15cm}}
\hline
{Step} & Step Details\\ \hline
1 & Description \\
 & \begin{minipage}[t]{15cm}
{\footnotesize
Authenticate to the notebook aspect of the LSST Science Platform
(NB-LSP). ~This is currently at
https://lsst-lsp-stable.ncsa.illinois.edu/nb.

\medskip }
\end{minipage}
\\ \cdashline{2-2}


 & Expected Result \\
 & \begin{minipage}[t]{15cm}{\footnotesize
Redirection to the spawner page of the NB-LSP allowing selection of the
containerized stack version and machine flavor.

\medskip }
\end{minipage} \\ \cdashline{2-2}

 & Actual Result \\
 & \begin{minipage}[t]{15cm}{\footnotesize

\medskip }
\end{minipage} \\ \cdashline{2-2}

 & Status: \textbf{ Not Executed } \\ \hline

2 & Description \\
 & \begin{minipage}[t]{15cm}
{\footnotesize
Spawn a container by:\\
1) choosing an appropriate stack version: e.g. the latest weekly.\\
2) choosing an appropriate machine flavor: e.g. medium\\
3) click ``Spawn''

\medskip }
\end{minipage}
\\ \cdashline{2-2}


 & Expected Result \\
 & \begin{minipage}[t]{15cm}{\footnotesize
Redirection to the JupyterLab environment served from the chosen
container containing the correct stack version.

\medskip }
\end{minipage} \\ \cdashline{2-2}

 & Actual Result \\
 & \begin{minipage}[t]{15cm}{\footnotesize

\medskip }
\end{minipage} \\ \cdashline{2-2}

 & Status: \textbf{ Not Executed } \\ \hline

3 & Description \\
 & \begin{minipage}[t]{15cm}
{\footnotesize
Open a new launcher by navigating in the top menu bar ``File''
-\textgreater{} ``New Launcher''

\medskip }
\end{minipage}
\\ \cdashline{2-2}


 & Expected Result \\
 & \begin{minipage}[t]{15cm}{\footnotesize
A launcher window with several sections, potentially with several kernel
versions for each.

\medskip }
\end{minipage} \\ \cdashline{2-2}

 & Actual Result \\
 & \begin{minipage}[t]{15cm}{\footnotesize

\medskip }
\end{minipage} \\ \cdashline{2-2}

 & Status: \textbf{ Not Executed } \\ \hline

4 & Description \\
 & \begin{minipage}[t]{15cm}
{\footnotesize
Select the option under ``Notebook'' labeled ``LSST'' by clicking on the
icon.

\medskip }
\end{minipage}
\\ \cdashline{2-2}


 & Expected Result \\
 & \begin{minipage}[t]{15cm}{\footnotesize
An empty notebook with a single empty cell. ~The kernel show up as
``LSST'' in the top right of the notebook.

\medskip }
\end{minipage} \\ \cdashline{2-2}

 & Actual Result \\
 & \begin{minipage}[t]{15cm}{\footnotesize

\medskip }
\end{minipage} \\ \cdashline{2-2}

 & Status: \textbf{ Not Executed } \\ \hline

5 & Description \\
 & \begin{minipage}[t]{15cm}
{\footnotesize
Execute a query in a notebook to select a small number of stars. In the
example code below, we query the WISE catalog, then extract the results
to an Astropy table.

\medskip }
\end{minipage}
\\ \cdashline{2-2}

 & Example Code \\
 & \begin{minipage}[t]{15cm}{\footnotesize
\begin{verbatim}
import pandas
import pyvo
service = pyvo.dal.TAPService('http://lsst-lsp-stable.ncsa.illinois.edu/api/tap')
\end{verbatim}

results = service.search(``SELECT ra, decl, w1mpro\_ep, w2mpro\_ep,
w3mpro\_ep FROM wise\_00.allwise\_p3as\_mep WHERE CONTAINS(POINT('ICRS',
ra, decl), CIRCLE('ICRS', 192.85, 27.13, .2)) = 1'')\\
tab = results.to\_table()

\medskip }
\end{minipage} \\ \cdashline{2-2}

 & Expected Result \\
 & \begin{minipage}[t]{15cm}{\footnotesize

\medskip }
\end{minipage} \\ \cdashline{2-2}

 & Actual Result \\
 & \begin{minipage}[t]{15cm}{\footnotesize

\medskip }
\end{minipage} \\ \cdashline{2-2}

 & Status: \textbf{ Not Executed } \\ \hline

6 & Description \\
 & \begin{minipage}[t]{15cm}
{\footnotesize
Using the example code below, save the files to your storage space on
the LSP Notebook Aspect.\\[2\baselineskip]Confirm that non-empty output
files appear on disk.

\medskip }
\end{minipage}
\\ \cdashline{2-2}

 & Example Code \\
 & \begin{minipage}[t]{15cm}{\footnotesize
tab.write('test.csv', format='ascii.csv')\\
tab.write('test.vot', format='votable')\\
tab.write('test.fits', format='fits')

\medskip }
\end{minipage} \\ \cdashline{2-2}

 & Expected Result \\
 & \begin{minipage}[t]{15cm}{\footnotesize
For the example given here, there should be the following files with the
file size as listed:

\begin{itemize}
\tightlist
\item
  test.csv 5.7M
\item
  test.vot 16M
\item
  test.fits 4.5M
\end{itemize}

\medskip }
\end{minipage} \\ \cdashline{2-2}

 & Actual Result \\
 & \begin{minipage}[t]{15cm}{\footnotesize

\medskip }
\end{minipage} \\ \cdashline{2-2}

 & Status: \textbf{ Not Executed } \\ \hline

7 & Description \\
 & \begin{minipage}[t]{15cm}
{\footnotesize
Check that these files contain the same number of rows:

\medskip }
\end{minipage}
\\ \cdashline{2-2}

 & Example Code \\
 & \begin{minipage}[t]{15cm}{\footnotesize
from astropy.table import Table\\
dat\_csv = Table.read('test.csv', format='ascii.csv')\\
dat\_vot = Table.read('test.vot', format='votable')\\
dat\_fits = Table.read('test.fits',
format='fits')\\[2\baselineskip]import numpy as np\\
print(np.size(dat\_csv), np.size(dat\_vot), np.size(dat\_fits))

\medskip }
\end{minipage} \\ \cdashline{2-2}

 & Expected Result \\
 & \begin{minipage}[t]{15cm}{\footnotesize
Print statement produces output ``97058 97058 97058''.

\medskip }
\end{minipage} \\ \cdashline{2-2}

 & Actual Result \\
 & \begin{minipage}[t]{15cm}{\footnotesize

\medskip }
\end{minipage} \\ \cdashline{2-2}

 & Status: \textbf{ Not Executed } \\ \hline

8 & Description \\
 & \begin{minipage}[t]{15cm}
{\footnotesize
Under the `File' menu at the top of your Jupyter notebook session,
select one of the following:\\[2\baselineskip]

\begin{itemize}
\tightlist
\item
  Save All, Exit, and Log Out
\item
  Exit and Log Out Without Saving
\end{itemize}

\medskip }
\end{minipage}
\\ \cdashline{2-2}


 & Expected Result \\
 & \begin{minipage}[t]{15cm}{\footnotesize
You will be returned to the LSP landing page:
\url{https://lsst-lsp-stable.ncsa.illinois.edu/} It is now safe to close
the browser window.~

\medskip }
\end{minipage} \\ \cdashline{2-2}

 & Actual Result \\
 & \begin{minipage}[t]{15cm}{\footnotesize

\medskip }
\end{minipage} \\ \cdashline{2-2}

 & Status: \textbf{ Not Executed } \\ \hline

\end{longtable}

\paragraph{ LVV-T45 - Verify implementation of Prompt Processing Data Quality Report
Definition }\mbox{}\\

Version \textbf{1}.
Open  \href{https://jira.lsstcorp.org/secure/Tests.jspa#/testCase/LVV-T45}{\textit{ LVV-T45 } }
test case in Jira.

Verify that the DMS produces a Prompt Processing Data Quality Report.
~Specifically check absolute value and temporal variation of\\
1. Photometric zeropoint\\
2. Sky brightness\\
3. Seeing\\
4. PSF\\
5. Detection efficiency

\textbf{ Preconditions}:\\


Execution status: {\bf Not Executed }

Final comment:\\


Detailed steps results:

\begin{longtable}{p{1cm}p{15cm}}
\hline
{Step} & Step Details\\ \hline
1 & Description \\
 & \begin{minipage}[t]{15cm}
{\footnotesize
Ingest raw data from L1 Test Stand DAQ.

\medskip }
\end{minipage}
\\ \cdashline{2-2}


 & Expected Result \\
 & \begin{minipage}[t]{15cm}{\footnotesize

\medskip }
\end{minipage} \\ \cdashline{2-2}

 & Actual Result \\
 & \begin{minipage}[t]{15cm}{\footnotesize

\medskip }
\end{minipage} \\ \cdashline{2-2}

 & Status: \textbf{ Not Executed } \\ \hline

2 & Description \\
 & \begin{minipage}[t]{15cm}
{\footnotesize
Perform the steps of Alert Production (including, but not necessarily
limited to, single frame processing, ISR, source detection/measurement,
PSF estimation, photometric and astrometric calibration, difference
imaging, DIASource detection/measurement, source association). During
Operations, it is presumed that these are automated for a given
dataset.~

\medskip }
\end{minipage}
\\ \cdashline{2-2}


 & Expected Result \\
 & \begin{minipage}[t]{15cm}{\footnotesize
An output dataset including difference images and DIASource and
DIAObject measurements.

\medskip }
\end{minipage} \\ \cdashline{2-2}

 & Actual Result \\
 & \begin{minipage}[t]{15cm}{\footnotesize

\medskip }
\end{minipage} \\ \cdashline{2-2}

 & Status: \textbf{ Not Executed } \\ \hline

3 & Description \\
 & \begin{minipage}[t]{15cm}
{\footnotesize
Verify that the expected data products have been produced, and that
catalogs contain reasonable values for measured quantities of interest.

\medskip }
\end{minipage}
\\ \cdashline{2-2}


 & Expected Result \\
 & \begin{minipage}[t]{15cm}{\footnotesize

\medskip }
\end{minipage} \\ \cdashline{2-2}

 & Actual Result \\
 & \begin{minipage}[t]{15cm}{\footnotesize

\medskip }
\end{minipage} \\ \cdashline{2-2}

 & Status: \textbf{ Not Executed } \\ \hline

4 & Description \\
 & \begin{minipage}[t]{15cm}
{\footnotesize
Load the Prompt Processing QC reports, and observe that a dynamically
updated Data Quality Report has become available at the relevant UI.

\medskip }
\end{minipage}
\\ \cdashline{2-2}


 & Expected Result \\
 & \begin{minipage}[t]{15cm}{\footnotesize
A Prompt Processing QC report is available via a UI, and contains
information about the photometric zeropoint, sky brightness, seeing,
PSF, and detection efficiency, and possibly other relevant quantities.

\medskip }
\end{minipage} \\ \cdashline{2-2}

 & Actual Result \\
 & \begin{minipage}[t]{15cm}{\footnotesize

\medskip }
\end{minipage} \\ \cdashline{2-2}

 & Status: \textbf{ Not Executed } \\ \hline

5 & Description \\
 & \begin{minipage}[t]{15cm}
{\footnotesize
Check that a static report is created and archived in a
readily-accessible location.

\medskip }
\end{minipage}
\\ \cdashline{2-2}


 & Expected Result \\
 & \begin{minipage}[t]{15cm}{\footnotesize
Persistence of a static QC report in an accessible location, containing
the same information as in the report from Step 3.

\medskip }
\end{minipage} \\ \cdashline{2-2}

 & Actual Result \\
 & \begin{minipage}[t]{15cm}{\footnotesize

\medskip }
\end{minipage} \\ \cdashline{2-2}

 & Status: \textbf{ Not Executed } \\ \hline

\end{longtable}

\paragraph{ LVV-T146 - Verify implementation of DMS Initialization Component }\mbox{}\\

Version \textbf{1}.
Open  \href{https://jira.lsstcorp.org/secure/Tests.jspa#/testCase/LVV-T146}{\textit{ LVV-T146 } }
test case in Jira.

Demonstrate that the DMS can be initialized in a safe state that will
not allow data corruption/loss.

\textbf{ Preconditions}:\\


Execution status: {\bf Not Executed }

Final comment:\\


Detailed steps results:

\begin{longtable}{p{1cm}p{15cm}}
\hline
{Step} & Step Details\\ \hline
1 & Description \\
 & \begin{minipage}[t]{15cm}
{\footnotesize
Power-cycle all of the DM systems at each Facility.

\medskip }
\end{minipage}
\\ \cdashline{2-2}


 & Expected Result \\
 & \begin{minipage}[t]{15cm}{\footnotesize
Restart of all DM systems.

\medskip }
\end{minipage} \\ \cdashline{2-2}

 & Actual Result \\
 & \begin{minipage}[t]{15cm}{\footnotesize

\medskip }
\end{minipage} \\ \cdashline{2-2}

 & Status: \textbf{ Not Executed } \\ \hline

2 & Description \\
 & \begin{minipage}[t]{15cm}
{\footnotesize
Observe each system and ensure that it has recovered in a properly
initialized state.

\medskip }
\end{minipage}
\\ \cdashline{2-2}


 & Expected Result \\
 & \begin{minipage}[t]{15cm}{\footnotesize
Systems are all active and initialized for their designated purpose.

\medskip }
\end{minipage} \\ \cdashline{2-2}

 & Actual Result \\
 & \begin{minipage}[t]{15cm}{\footnotesize

\medskip }
\end{minipage} \\ \cdashline{2-2}

 & Status: \textbf{ Not Executed } \\ \hline

\end{longtable}

\paragraph{ LVV-T144 - Verify implementation of Task Specification }\mbox{}\\

Version \textbf{1}.
Open  \href{https://jira.lsstcorp.org/secure/Tests.jspa#/testCase/LVV-T144}{\textit{ LVV-T144 } }
test case in Jira.

Verify that the DMS provides the ability to define a new or modified
pipeline task without recompilation.

\textbf{ Preconditions}:\\


Execution status: {\bf Not Executed }

Final comment:\\


Detailed steps results:

\begin{longtable}{p{1cm}p{15cm}}
\hline
{Step} & Step Details\\ \hline
1 & Description \\
 & \begin{minipage}[t]{15cm}
{\footnotesize
Inspect software architecture. ~Verify that there exist Tasks that can
be run and configured without re-compilation.

\medskip }
\end{minipage}
\\ \cdashline{2-2}


 & Expected Result \\
 & \begin{minipage}[t]{15cm}{\footnotesize
Confirmation that the software architecture has allowed for
reconfiguring and running Tasks without recompilation.

\medskip }
\end{minipage} \\ \cdashline{2-2}

 & Actual Result \\
 & \begin{minipage}[t]{15cm}{\footnotesize

\medskip }
\end{minipage} \\ \cdashline{2-2}

 & Status: \textbf{ Not Executed } \\ \hline

2 & Description \\
 & \begin{minipage}[t]{15cm}
{\footnotesize
Verify that an example science algorithm can be run through one of these
Tasks.~ Three examples from different areas: source measurement, image
subtraction, and photometric-redshift estimation.

\medskip }
\end{minipage}
\\ \cdashline{2-2}


 & Expected Result \\
 & \begin{minipage}[t]{15cm}{\footnotesize
Successful Task execution with different configurations, including
confirmation that the outputs are different from tasks with altered
configurations.

\medskip }
\end{minipage} \\ \cdashline{2-2}

 & Actual Result \\
 & \begin{minipage}[t]{15cm}{\footnotesize

\medskip }
\end{minipage} \\ \cdashline{2-2}

 & Status: \textbf{ Not Executed } \\ \hline

\end{longtable}

\paragraph{ LVV-T145 - Verify implementation of Task Configuration }\mbox{}\\

Version \textbf{1}.
Open  \href{https://jira.lsstcorp.org/secure/Tests.jspa#/testCase/LVV-T145}{\textit{ LVV-T145 } }
test case in Jira.

Verify that the DMS software provides configuration control to define,
override, and verify the configuration for a DMS Task.

\textbf{ Preconditions}:\\


Execution status: {\bf Not Executed }

Final comment:\\


Detailed steps results:

\begin{longtable}{p{1cm}p{15cm}}
\hline
{Step} & Step Details\\ \hline
1 & Description \\
 & \begin{minipage}[t]{15cm}
{\footnotesize
Inspect software design to verify that one can define the configuration
for a Task.

\medskip }
\end{minipage}
\\ \cdashline{2-2}


 & Expected Result \\
 & \begin{minipage}[t]{15cm}{\footnotesize

\medskip }
\end{minipage} \\ \cdashline{2-2}

 & Actual Result \\
 & \begin{minipage}[t]{15cm}{\footnotesize

\medskip }
\end{minipage} \\ \cdashline{2-2}

 & Status: \textbf{ Not Executed } \\ \hline

2 & Description \\
 & \begin{minipage}[t]{15cm}
{\footnotesize
Run a Task with a known invalid configuration. ~Verify that the error is
caught before the science algorithm executes.

\medskip }
\end{minipage}
\\ \cdashline{2-2}


 & Expected Result \\
 & \begin{minipage}[t]{15cm}{\footnotesize

\medskip }
\end{minipage} \\ \cdashline{2-2}

 & Actual Result \\
 & \begin{minipage}[t]{15cm}{\footnotesize

\medskip }
\end{minipage} \\ \cdashline{2-2}

 & Status: \textbf{ Not Executed } \\ \hline

3 & Description \\
 & \begin{minipage}[t]{15cm}
{\footnotesize
Run a simple task with two different configurations that make a material
difference for a Task. ~E.g., specify a different source detection
threshold. ~Verify that the configuration is different between the two
runs through difference in recorded provenance and in results.

\medskip }
\end{minipage}
\\ \cdashline{2-2}


 & Expected Result \\
 & \begin{minipage}[t]{15cm}{\footnotesize

\medskip }
\end{minipage} \\ \cdashline{2-2}

 & Actual Result \\
 & \begin{minipage}[t]{15cm}{\footnotesize

\medskip }
\end{minipage} \\ \cdashline{2-2}

 & Status: \textbf{ Not Executed } \\ \hline

\end{longtable}

\paragraph{ LVV-T1264 - Verify implementation of archiving camera test data }\mbox{}\\

Version \textbf{1}.
Open  \href{https://jira.lsstcorp.org/secure/Tests.jspa#/testCase/LVV-T1264}{\textit{ LVV-T1264 } }
test case in Jira.

Verify that a subset of camera test data has been ingested into Butler
repos and is available through standard data access tools.

\textbf{ Preconditions}:\\


Execution status: {\bf Not Executed }

Final comment:\\


Detailed steps results:

\begin{longtable}{p{1cm}p{15cm}}
\hline
{Step} & Step Details\\ \hline
1 & Description \\
 & \begin{minipage}[t]{15cm}
{\footnotesize
Obtain some data on a camera test stand.

\medskip }
\end{minipage}
\\ \cdashline{2-2}


 & Expected Result \\
 & \begin{minipage}[t]{15cm}{\footnotesize

\medskip }
\end{minipage} \\ \cdashline{2-2}

 & Actual Result \\
 & \begin{minipage}[t]{15cm}{\footnotesize

\medskip }
\end{minipage} \\ \cdashline{2-2}

 & Status: \textbf{ Not Executed } \\ \hline

2 & Description \\
 & \begin{minipage}[t]{15cm}
{\footnotesize
Wait a sufficient amount of time, then confirm that automatic
transfer/ingest of the data has occurred, and a repo is available at
NCSA.

\medskip }
\end{minipage}
\\ \cdashline{2-2}


 & Expected Result \\
 & \begin{minipage}[t]{15cm}{\footnotesize
The data is present at NCSA in non-empty repos.

\medskip }
\end{minipage} \\ \cdashline{2-2}

 & Actual Result \\
 & \begin{minipage}[t]{15cm}{\footnotesize

\medskip }
\end{minipage} \\ \cdashline{2-2}

 & Status: \textbf{ Not Executed } \\ \hline

3 & Description \\
 & \begin{minipage}[t]{15cm}
{\footnotesize
Identify the relevant Butler repo of ingested camera test stand data.

\medskip }
\end{minipage}
\\ \cdashline{2-2}


 & Expected Result \\
 & \begin{minipage}[t]{15cm}{\footnotesize

\medskip }
\end{minipage} \\ \cdashline{2-2}

 & Actual Result \\
 & \begin{minipage}[t]{15cm}{\footnotesize

\medskip }
\end{minipage} \\ \cdashline{2-2}

 & Status: \textbf{ Not Executed } \\ \hline

4 & Description \\
 & \begin{minipage}[t]{15cm}
{\footnotesize
Identify the path to the data repository, which we will refer to as
`DATA/path', then execute the following:

\medskip }
\end{minipage}
\\ \cdashline{2-2}

 & Example Code \\
 & \begin{minipage}[t]{15cm}{\footnotesize
\begin{verbatim}
import lsst.daf.persistence as dafPersist
butler = dafPersist.Butler(inputs='DATA/path')
\end{verbatim}

\medskip }
\end{minipage} \\ \cdashline{2-2}

 & Expected Result \\
 & \begin{minipage}[t]{15cm}{\footnotesize
Butler repo available for reading.

\medskip }
\end{minipage} \\ \cdashline{2-2}

 & Actual Result \\
 & \begin{minipage}[t]{15cm}{\footnotesize

\medskip }
\end{minipage} \\ \cdashline{2-2}

 & Status: \textbf{ Not Executed } \\ \hline

5 & Description \\
 & \begin{minipage}[t]{15cm}
{\footnotesize
Read various repo data products with the Butler, and confirm that they
contain the expected data.

\medskip }
\end{minipage}
\\ \cdashline{2-2}


 & Expected Result \\
 & \begin{minipage}[t]{15cm}{\footnotesize
Camera test stand data that is well-formed.

\medskip }
\end{minipage} \\ \cdashline{2-2}

 & Actual Result \\
 & \begin{minipage}[t]{15cm}{\footnotesize

\medskip }
\end{minipage} \\ \cdashline{2-2}

 & Status: \textbf{ Not Executed } \\ \hline

\end{longtable}



\newpage
\appendix
% Make sure lsst-texmf/bin/generateAcronyms.py is in your path
\section{Acronyms used in this document}\label{sec:acronyms}
\input{acronyms.tex}

\newpage

% generated from JIRA project LVV
% using template at /usr/local/lib/python3.7/site-packages/docsteady/templates/dm-tpr-appendix.latex.jinja2.
% using docsteady version 1.2rc24
% Please do not edit -- update information in Jira instead

\section{Traceability}

\begin{longtable}{p{3cm}p{3cm}L{9cm}}
\hline
\textbf{Test Case} & \textbf{VE Key} & \textbf{VE Summary} \\ \hline
\href{https://jira.lsstcorp.org/secure/Tests.jspa#/testCase/LVV-T28}{LVV-T28} &
  \href{https://jira.lsstcorp.org/browse/LVV-178}{LVV-178}
  & DMS-REQ-0347-V-01: Measurements in catalogs
 \\ \cdashline{2-3}
\hline
\href{https://jira.lsstcorp.org/secure/Tests.jspa#/testCase/LVV-T36}{LVV-T36} &
  \href{https://jira.lsstcorp.org/browse/LVV-7}{LVV-7}
  & DMS-REQ-0010-V-01: Difference Exposures
 \\ \cdashline{2-3}
\hline
\href{https://jira.lsstcorp.org/secure/Tests.jspa#/testCase/LVV-T38}{LVV-T38} &
  \href{https://jira.lsstcorp.org/browse/LVV-29}{LVV-29}
  & DMS-REQ-0069-V-01: Processed Visit Images
 \\ \cdashline{2-3}
\hline
\href{https://jira.lsstcorp.org/secure/Tests.jspa#/testCase/LVV-T39}{LVV-T39} &
  \href{https://jira.lsstcorp.org/browse/LVV-12}{LVV-12}
  & DMS-REQ-0029-V-01: Generate Photometric Zeropoint for Visit Image
 \\ \cdashline{2-3}
\hline
\href{https://jira.lsstcorp.org/secure/Tests.jspa#/testCase/LVV-T40}{LVV-T40} &
  \href{https://jira.lsstcorp.org/browse/LVV-13}{LVV-13}
  & DMS-REQ-0030-V-01: Absolute accuracy of WCS
 \\ \cdashline{2-3}
\hline
\href{https://jira.lsstcorp.org/secure/Tests.jspa#/testCase/LVV-T42}{LVV-T42} &
  \href{https://jira.lsstcorp.org/browse/LVV-31}{LVV-31}
  & DMS-REQ-0072-V-01: Processed Visit Image Content
 \\ \cdashline{2-3}
\hline
\href{https://jira.lsstcorp.org/secure/Tests.jspa#/testCase/LVV-T45}{LVV-T45} &
  \href{https://jira.lsstcorp.org/browse/LVV-39}{LVV-39}
  & DMS-REQ-0097-V-01: Level 1 Data Quality Report Definition
 \\ \cdashline{2-3}
\hline
\href{https://jira.lsstcorp.org/secure/Tests.jspa#/testCase/LVV-T46}{LVV-T46} &
  \href{https://jira.lsstcorp.org/browse/LVV-41}{LVV-41}
  & DMS-REQ-0099-V-01: Level 1 Performance Report Definition
 \\ \cdashline{2-3}
\hline
\href{https://jira.lsstcorp.org/secure/Tests.jspa#/testCase/LVV-T125}{LVV-T125} &
  \href{https://jira.lsstcorp.org/browse/LVV-6}{LVV-6}
  & DMS-REQ-0009-V-01: Simulated Data
 \\ \cdashline{2-3}
\hline
\href{https://jira.lsstcorp.org/secure/Tests.jspa#/testCase/LVV-T126}{LVV-T126} &
  \href{https://jira.lsstcorp.org/browse/LVV-14}{LVV-14}
  & DMS-REQ-0032-V-01: Image Differencing
 \\ \cdashline{2-3}
\hline
\href{https://jira.lsstcorp.org/secure/Tests.jspa#/testCase/LVV-T133}{LVV-T133} &
  \href{https://jira.lsstcorp.org/browse/LVV-182}{LVV-182}
  & DMS-REQ-0351-V-01: Provide Beam Projector Coordinate Calculation
Software
 \\ \cdashline{2-3}
\hline
\href{https://jira.lsstcorp.org/secure/Tests.jspa#/testCase/LVV-T144}{LVV-T144} &
  \href{https://jira.lsstcorp.org/browse/LVV-136}{LVV-136}
  & DMS-REQ-0305-V-01: Task Specification
 \\ \cdashline{2-3}
\hline
\href{https://jira.lsstcorp.org/secure/Tests.jspa#/testCase/LVV-T145}{LVV-T145} &
  \href{https://jira.lsstcorp.org/browse/LVV-137}{LVV-137}
  & DMS-REQ-0306-V-01: Task Configuration
 \\ \cdashline{2-3}
\hline
\href{https://jira.lsstcorp.org/secure/Tests.jspa#/testCase/LVV-T146}{LVV-T146} &
  \href{https://jira.lsstcorp.org/browse/LVV-128}{LVV-128}
  & DMS-REQ-0297-V-01: DMS Initialization Component
 \\ \cdashline{2-3}
\hline
\href{https://jira.lsstcorp.org/secure/Tests.jspa#/testCase/LVV-T151}{LVV-T151} &
  \href{https://jira.lsstcorp.org/browse/LVV-35}{LVV-35}
  & DMS-REQ-0078-V-01: Catalog Export Formats
 \\ \cdashline{2-3}
\hline
\href{https://jira.lsstcorp.org/secure/Tests.jspa#/testCase/LVV-T1232}{LVV-T1232} &
  \href{https://jira.lsstcorp.org/browse/LVV-35}{LVV-35}
  & DMS-REQ-0078-V-01: Catalog Export Formats
 \\ \cdashline{2-3}
\hline
\href{https://jira.lsstcorp.org/secure/Tests.jspa#/testCase/LVV-T1264}{LVV-T1264} &
  \href{https://jira.lsstcorp.org/browse/LVV-9637}{LVV-9637}
  & DMS-REQ-0372-V-01: Archiving Camera Test Data
 \\ \cdashline{2-3}
\hline
\href{https://jira.lsstcorp.org/secure/Tests.jspa#/testCase/LVV-T1756}{LVV-T1756} &
  \href{https://jira.lsstcorp.org/browse/LVV-3401}{LVV-3401}
  & DMS-REQ-0359-V-01: RMS photometric repeatability in uzy
 \\ \cdashline{2-3}
\hline
\href{https://jira.lsstcorp.org/secure/Tests.jspa#/testCase/LVV-T1757}{LVV-T1757} &
  \href{https://jira.lsstcorp.org/browse/LVV-9759}{LVV-9759}
  & DMS-REQ-0359-V-10: RMS photometric repeatability in gri
 \\ \cdashline{2-3}
\hline
\href{https://jira.lsstcorp.org/secure/Tests.jspa#/testCase/LVV-T1758}{LVV-T1758} &
  \href{https://jira.lsstcorp.org/browse/LVV-9758}{LVV-9758}
  & DMS-REQ-0359-V-09: Repeatability outlier limit in uzy
 \\ \cdashline{2-3}
 &   \href{https://jira.lsstcorp.org/browse/LVV-9752}{LVV-9752}
  & DMS-REQ-0359-V-03: Max fraction of outliers among non-saturated sources
 \\ \cdashline{2-3}
\hline
\href{https://jira.lsstcorp.org/secure/Tests.jspa#/testCase/LVV-T1759}{LVV-T1759} &
  \href{https://jira.lsstcorp.org/browse/LVV-9752}{LVV-9752}
  & DMS-REQ-0359-V-03: Max fraction of outliers among non-saturated sources
 \\ \cdashline{2-3}
 &   \href{https://jira.lsstcorp.org/browse/LVV-9754}{LVV-9754}
  & DMS-REQ-0359-V-05: Repeatability outlier limit in gri
 \\ \cdashline{2-3}
\hline
\href{https://jira.lsstcorp.org/secure/Tests.jspa#/testCase/LVV-T1946}{LVV-T1946} &
  \href{https://jira.lsstcorp.org/browse/LVV-178}{LVV-178}
  & DMS-REQ-0347-V-01: Measurements in catalogs
 \\ \cdashline{2-3}
\hline
\href{https://jira.lsstcorp.org/secure/Tests.jspa#/testCase/LVV-T1947}{LVV-T1947} &
  \href{https://jira.lsstcorp.org/browse/LVV-178}{LVV-178}
  & DMS-REQ-0347-V-01: Measurements in catalogs
 \\ \cdashline{2-3}
\hline
\end{longtable}


\end{document}
